\section{Introduction}\label{sec:intro}

Structure editors let you directly edit the tree structure of your program.
By eliminating syntax errors, they
  provide cognitive benefits for novices and end-users;
  simplify language composition and language-aware editor tooling;
  and form an essential component of truly continuous live programming.
\note{change phrasing to indicate longstanding problem: "structure editors have long promised..."}

\begin{itemize}
\item but usability issues
\begin{itemize}
    \item Despite these myriad benefits, however, structure editors have failed to achieve
      widespread adoption by programmers beyond the novice level.
    \item early reputation of being rigid, inefficient
    \item some progress on relaxing rigid out-in construction
\end{itemize}
\item less attention paid to restructuring existing code
\begin{itemize}
  \item contemporary structure editors continue to suffer from high viscosity
  \item eg Scratch
  \begin{itemize}
    \item requires clicking and dragging, large visual real estate
    \item wrapping existing blocks in a new block:
      create the new block,
      pull out existing blocks and into new block,
      then re-insert new block
    \item deleting a form while preserving some or all children:
      need to pull child out of form before deleting,
      then pull child back into the form
    \item replacing a block with a different one
    \item \note{
      perhaps modify framing to cite papers that already note structure editing is a problem, take quotes, establish as recognized problem.
      look at Guzdial's paper.
    }
  \end{itemize}
  \item eg MPS
  \begin{itemize}
    \item designed to more closely emulate text editing experience
    \item better affordances for some edits described above,
      eg linear text-like entry of syntactic forms,
      eg auto-preserving first child on deletion of form (if same sort)
    \item but text-like appearance presents new challenges,
      eg selecting \texttt{2 + 3} in \texttt{2 + 3 * 4},
      eg lack of signifiers making it difficult to predict deletion (experts struggle
  \end{itemize}
  \item in general, no ability to select and directly modify substructural
    and cross-cutting portions of programs (where desirable selections are
    dictated not by the tree structure of language grammar but rather the
    projection of the program into a 2d layout)
  \item \note{perhaps describe more generally as tension between linear vs tree structures}
  \item \note{distinguish semantic structure vs surface structure? syntactic structure? user structure? projectional structure?}
\end{itemize}
\item our approach
\begin{itemize}
    \item \note{tree and linear structure can co-exist}
    \item we propose a new model of tiles for representing a structure editor's edit state along with editing mechanics for these tiles that...
\end{itemize}
\item contributions
\begin{itemize}
  \item \note{mention modal nature as a way of doing edits within projectional structure}
  \item \note{note that mode ensures the only way to exit is in a semantically structured state}
\end{itemize}
\item outline
\end{itemize}

