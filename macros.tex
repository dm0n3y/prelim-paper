% !TEX root = prelim-paper.tex

% \newcommand{\Label}[1]{\vspace{-20px}\label{#1}%
%   {\small\textcolor{cyan}{(\texttt{#1})}}\vspace{20px}%
% }

\newcommand{\note}[1]{\textcolor{blue}{#1}}

\newcommand{\tylr}{\texttt{tylr}}

% Violet hot dog buns; highlight color helps distinguish them
\newcommand{\llparenthesiscolor}{\textcolor{violet}{\llparenthesis}}
\newcommand{\rrparenthesiscolor}{\textcolor{violet}{\rrparenthesis}}

\newcommand{\styp}{t}
\newcommand{\spat}{p}
\newcommand{\sexp}{e}
\newcommand{\shole}{\llparenthesiscolor\rrparenthesiscolor}
\newcommand{\svar}[1]{#1}
\newcommand{\snum}{\texttt{num}}
\newcommand{\sbool}{\texttt{bool}}
\newcommand{\sarr}[2]{#1 \rightarrow #2}
\newcommand{\sprod}[2]{#1\texttt{,}#2}
\newcommand{\sann}[2]{#1~\texttt{:}~#2}
\newcommand{\sboollit}[1]{#1}
\newcommand{\snumlit}[1]{#1}
\newcommand{\sparen}[1]{\texttt{(}#1\texttt{)}}
\newcommand{\slam}[2]{\lambda #1.#2}
\newcommand{\slet}[3]{\texttt{let}~#1~\texttt{=}~#2~\texttt{in}~#3}
\newcommand{\splus}[2]{#1~\texttt{+}~#2}
\newcommand{\smult}[2]{#1~\texttt{*}~#2}
\newcommand{\sequals}[2]{#1~\texttt{==}~#2}
\newcommand{\sap}[2]{#1\texttt{\textvisiblespace}#2}
\newcommand{\scond}[3]{#1~\texttt{?}~#2~\texttt{:}~#3}

\newcommand{\action}{\alpha}
\newcommand{\editState}{\mathcal{E}}
\newcommand{\performAction}[3]{#1 \xlongrightarrow{#2} #3}

\newcommand{\tiles}{T}
\newcommand{\tile}{\tau}
\newcommand{\token}{\kappa}
\newcommand{\tokenLit}[1]{\underline{#1}}
\newcommand{\selection}{\Psi}
\newcommand{\selem}{\psi}
\newcommand{\tframe}{\Phi}
\newcommand{\tfrelem}{\phi}
\newcommand{\tframelit}[2]{#1 \gg #2}

\newcommand{\zip}{\zeta}
\newcommand{\zipper}[2]{#1 \gtrdot #2}
\newcommand{\subject}{\sigma}
\newcommand{\pointing}[2]{#1\hspace{-3pt}\multimapbothvert\hspace{-3pt}#2}
\newcommand{\selecting}[3]{#1\overline{#2}#3}
\newcommand{\restructuring}[3]{#1\widehat{#2}#3}

\newcommand{\typ}{\texttt{typ}}
\newcommand{\pat}{\texttt{pat}}
\newcommand{\expr}{\texttt{exp}}

\newcommand{\disassembleTile}[1]{\lfloor~#1~\rfloor}
\newcommand{\stepDisassembleSelection}[2]{#1 \searrow #2}
\newcommand{\disassembleSelection}[2]{#1 \succeq #2}


\newcommand{\cmttclo}[2]{\mathsf{clo}(#1, #2)}

% \newcommand{\CaptionLabel}[2]{
%   \caption{#1 {\small\textcolor{cyan}{(#2)}}}
%   \label{#2}}
\newcommand{\CaptionLabel}[2]{
  \caption{#1}
  \label{#2}}

% \newcommand{\llparenthesiscolor}{\textcolor{red}{\lfloor}}
% \newcommand{\rrparenthesiscolor}{\textcolor{red}{\rfloor}}

\newcommand{\fmap} [3] {\{ #1 ~ \rotatebox[origin=c]{180}{$\Lsh$} ~ #3 \}_{#1 \in #2}}

%% TODO if feeling really obsessive, use the following in place of x,u,c,b
\newcommand{\varVar}{x}
\newcommand{\varHole}{u}
\newcommand{\econst}{c}
\newcommand{\tbase}{b}

% HTyp and HExp
\newcommand{\isComplete}[1]{#1~\mathsf{complete}}

% HTyp
\newcommand{\htyp}[0]{\tau} % Meta variable for types
\newcommand{\tarr}[2]{#1 \rightarrow #2}
%\newcommand{\tsum}[2]{#1 + #2}
\newcommand{\tprod}[2]{#1 \times #2}
\newcommand{\tunit}{\mathsf{1}}
\newcommand{\tnum}{\mathsf{num}}
\newcommand{\tb}{\texttt{b}}
\newcommand{\tehole}{\llparenthesiscolor\rrparenthesiscolor}
\newcommand{\tsum}[2]{{#1} + {#2}}
\newcommand{\trec}[2]{\mu(#1.#2)}

\newcommand{\tconsistent}[2]{#1 \sim #2}
\newcommand{\tinconsistent}[2]{#1 \nsim #2}
\newcommand{\tconsistentc}[2]{#1 \sim_c #2}

% Expression forms
\newcommand{\hexp}{e} % Hexp without palettes
\newcommand{\pexp}{p} % Meta variable for palette expressions
\newcommand{\encExp}{\mathtt{Exp}} % Whatever htyp encoded things should have

% HExp
\newcommand{\hlam}[2]{\lambda #1.#2}
\newcommand{\halam}[3]{\lambda #1{:}#2.#3}
\newcommand{\hap}[2]{#1~#2}
\newcommand{\hapP}[2]{(#1)~(#2)} % Extra paren around function term
\newcommand{\hpair}[2]{(#1, #2)}
\newcommand{\hprj}[2]{\mathsf{prj}_{#1}(#2)}
\newcommand{\hprl}[1]{\mathsf{prl}(#1)}
\newcommand{\hprr}[1]{\mathsf{prr}(#1)}
\newcommand{\htriv}{()}
\newcommand{\lblL}{\mathsf{L}}
\newcommand{\lblR}{\mathsf{R}}
\newcommand{\hnum}[1]{\underline{#1}}
\newcommand{\helet}[3]{\mathsf{let}~#1=#2~\mathsf{in}~#3}
%\newcommand{\hcase}[5]{\mathsf{case}\,#1\,\mathsf{of}\,#2\Rightarrow#3~\vert~#4\Rightarrow#5}
\newcommand{\hadd}[2]{#1 + #2}
\newcommand{\hehole}[1]{\llparenthesiscolor\rrparenthesiscolor^{#1}}
% \newcommand{\hhole}[1]{\setlength{\fboxsep}{0pt}\fcolorbox{red}{white}{\vphantom{)}$#1$}}
\newcommand{\hhole}[2]{\llparenthesiscolor#1\rrparenthesiscolor^{#2}}
% \newcommand{\hhole}[1]{
  % \setlength{\fboxsep}{0pt}
  % \colorbox{violet!10!white!100}{\ensuremath{\llparenthesiscolor#1\rrparenthesiscolor}}}
\newcommand{\hindet}[1]{\lceil#1\rceil}
%\newcommand{\hinj}[2]{\texttt{inj}_{#1}({#2})}
\newcommand{\hinL}[1]{\mathsf{inl}(#1)}
\newcommand{\hinR}[1]{\mathsf{inr}(#1)}
\newcommand{\hcase}[5]{\texttt{case}({#1},{#2}.{#3},{#4}.{#5})}
\newcommand{\hroll}[1]{\mathsf{roll}(#1)}
\newcommand{\hunroll}[1]{\mathsf{unroll}(#1)}
\newcommand{\livelitname}[1]{\$#1}
\newcommand{\haplivelit}[4]{\livelitname{#2}\langle #3; #4 \rangle^{#1}}
\newcommand{\hsplice}[2]{#1 : #2}

% Palettes
\newcommand{\pexpPalLet}[3]{\keyword{let palette}\,#1\,\keyword{=}\,#2\,\keyword{in}\,#3}
\newcommand{\pexpPalAp}[4]{#1 \left(#2;\,#4 : #3 \right)}
\newcommand{\pexpPalApF}[3]{#1 \left(#2;\,#3\right)}

% Palette definitions
\newcommand{\pDef}{\pi} % Meta variable for palette definitions
\newcommand{\pDefRecord}[3]{\{ \mathsf{expand}: #1, \, \mathsf{modelTyp}: #2, \, \mathsf{expandTyp}: #3\}}
\newcommand{\pDefRecordF}[4]{\{ \mathsf{expand}: #1, \, \mathsf{modelTyp}: #2, \, \mathsf{spliceTyp}: #3, \, \mathsf{expandTyp}: #4\}}
\newcommand{\pPhiWF}[1]{#1 \, \, \mathsf{palctx}}

\newcommand{\hGamma}{\Gamma}
\newcommand{\pPhi}{\Phi}
\newcommand{\EmptyhGamma}{\emptyset}
\newcommand{\EmptyDelta}{\emptyset}
\newcommand{\domof}[1]{\text{dom}(#1)}
\newcommand{\hsyn}[3]{#1 \vdash #2 \Rightarrow #3}
\newcommand{\hana}[3]{#1 \vdash #2 \Leftarrow #3}

% ZTyp and ZExp
\newcommand{\zlsel}[1]{{\bowtie}{#1}}
\newcommand{\zrsel}[1]{{#1}{\bowtie}}

%\newcommand{\zwsel}[1]{\adjustbox{cframe=blue}{\ensuremath{{\textcolor{blue}{\triangleright}}{#1}{\textcolor{blue}{\triangleleft}}}}}
\newcommand{\zwsel}[1]{
  \setlength{\fboxsep}{0pt}
  \colorbox{green!10!white!100}{
    \ensuremath{{{\textcolor{Green}{{\hspace{-2px}\triangleright}}}}{#1}{\textcolor{Green}{\triangleleft{\vphantom{\tehole}}}}}}
}
%\newcommand{\zwsel}[1]{{\triangleright}{#1}{\triangleleft}}

\newcommand{\removeSel}[1]{#1^{\diamond}}

% ZTyp
\newcommand{\ztau}{\hat{\tau}}

% ZExp
\newcommand{\zexp}{\hat{e}}

% Direction
\newcommand{\dParent}{\mathtt{parent}}
\newcommand{\dChild}{\mathtt{firstChild}}
\newcommand{\dNext}{\mathtt{nextSib}}
\newcommand{\dPrev}{\mathtt{prevSib}}

% Action
\newcommand{\aMove}[1]{\mathtt{move}~#1}
	\newcommand{\zrightmost}[1]{\mathsf{rightmost}(#1)}
	\newcommand{\zleftmost}[1]{\mathsf{leftmost}(#1)}
\newcommand{\aSelect}[1]{\mathtt{sel}~#1}
\newcommand{\aDel}{\mathtt{del}}
\newcommand{\aReplace}[1]{\mathtt{replace}~#1}
\newcommand{\aConstruct}[1]{\mathtt{construct}~#1}
\newcommand{\aConstructx}[1]{#1}
\newcommand{\aFinish}{\mathtt{finish}}

\newcommand{\performAna}[5]{#1 \vdash #2 \xlongrightarrow{#4} #5 \Leftarrow #3}
\newcommand{\performAnaI}[5]{#1 \vdash #2 \xlongrightarrow{#4}\hspace{-3px}{}^{*}~ #5 \Leftarrow #3}
\newcommand{\performSyn}[6]{#1 \vdash #2 \Rightarrow #3 \xlongrightarrow{#4} #5 \Rightarrow #6}
\newcommand{\performSynI}[6]{#1 \vdash #2 \Rightarrow #3 \xlongrightarrow{#4}\hspace{-3px}{}^{*}~ #5 \Rightarrow #6}
\newcommand{\performTyp}[3]{#1 \xlongrightarrow{#2} #3}
\newcommand{\performTypI}[3]{#1 \xlongrightarrow{#2}\hspace{-3px}{}^{*}~#3}

\newcommand{\performMove}[3]{#1 \xlongrightarrow{#2} #3}
\newcommand{\performDel}[2]{#1 \xlongrightarrow{\aDel} #2}

% Form
\newcommand{\farr}{\mathtt{arrow}}
\newcommand{\fnum}{\mathtt{num}}
\newcommand{\fsum}{\mathtt{sum}}

\newcommand{\fasc}{\mathtt{asc}}
\newcommand{\fvar}[1]{\mathtt{var}~#1}
\newcommand{\flam}[1]{\mathtt{lam}~#1}
\newcommand{\fap}{\mathtt{ap}}
\newcommand{\farg}{\mathtt{arg}}
\newcommand{\fnumlit}[1]{\mathtt{lit}~#1}
\newcommand{\fplus}{\mathtt{plus}}
\newcommand{\fhole}{\mathtt{hole}}
\newcommand{\fnehole}{\mathtt{nehole}}

\newcommand{\finj}[1]{\mathtt{inj}~#1}
\newcommand{\fcase}[2]{\mathtt{case}~#1~#2}

% Talk about formal rules in example
\newcommand{\refrule}[1]{\textrm{Rule~(#1)}}

\newcommand{\herase}[1]{\left|#1\right|_\textsf{erase}}

\newcommand{\arrmatch}[2]{#1 \blacktriangleright_{\rightarrow} #2}
%% TODO maybe write underbracket
%% \newcommand{\groundmatch}[2]{\underline{#1} = #2}
\newcommand{\groundmatch}[2]{#1 \blacktriangleright_{\mathsf{ground}} #2}
\newcommand{\prodmatch}[2]{#1 \blacktriangleright_{\times} #2}
\newcommand{\summatch}[2]{#1 \blacktriangleright_{+} #2}


\newcommand{\TABperformAna}[5]{#1 \vdash & #2                & \xlongrightarrow{#4} & #5 & \Leftarrow #3}
\newcommand{\TABperformSyn}[6]{#1 \vdash & #2 \Rightarrow #3 & \xlongrightarrow{#4} & #5 \Rightarrow #6}
\newcommand{\TABperformTyp}[3]{& #1 & \xlongrightarrow{#2} & #3}

\newcommand{\TABperformMove}[3]{#1 & \xlongrightarrow{#2} & #3}
\newcommand{\TABperformDel}[2]{#1 \xlongrightarrow{\aDel} #2}

\newcommand{\sumhasmatched}[2]{#1 \mathrel{\textcolor{black}{\blacktriangleright_{+}}} #2}

%%%% DYNAMICS %%%%
%% TODO remove these macros
%% marks for eval
\newcommand{\unevaled}{\times}
\newcommand{\evaled}{\checkmark}
\newcommand{\markname}{m}

\newcommand{\mvar}[0]{u}
\newcommand{\subst}[0]{\sigma}
\newcommand{\substitute}[3]{[#1/#2]#3}
\newcommand{\substitutesub}[4]{[#1/#2]_{#3}\,#4}
\newcommand{\fvof}[1]{\mathsf{FV}(#1)}
\newcommand{\dexp}[0]{d}
\newcommand{\dconst}[0]{c}
\newcommand{\dval}[0]{\ddot{v}}
%% TODO remove this macro
\newcommand{\dcast}[2]{\langle #1 \rangle ~ #2}
%% TODO make the following two look better
\newcommand{\dcasttwo}[3]{#1 \langle{#2}\Rightarrow{#3}\rangle}
\newcommand{\dcastthree}[4]
  {#1 \langle{#2}\Rightarrow{#3}\Rightarrow{#4}\rangle} %% sugared version
  %% {\dcasttwo{\dcasttwo{#1}{#2}{#3}}{#3}{#4}} %% unsugared version
\newcommand{\dcastfail}[3]{#1 \langle{#2}\Rightarrow{\tehole}\not\Rightarrow{#3}\rangle}
%% \newcommand{\dlam}[3]{\lambda #1:#2.#3}
\newcommand{\dlam}[3]{\halam{#1}{#2}{#3}}
\newcommand{\dap}[2]{#1(#2)}
\newcommand{\dapP}[2]{(#1)(#2)} % Extra paren around function term
\newcommand{\dnum}[1]{\underline{#1}}
%\newcommand{\dcase}[5]{\mathsf{case}\,#1\,\mathsf{of}\,#2\Rightarrow#3~\vert~#4\Rightarrow#5}
\newcommand{\dadd}[2]{#1 + #2}
%% TODO third arg should be empty
\newcommand{\dehole}[2]{{\llparenthesiscolor\rrparenthesiscolor}{^{#1}_{#2}}}
%% TODO fourth arg should be empty
\newcommand{\dhole}[4]{\leftidx{^{#4}}{\llparenthesiscolor#1\rrparenthesiscolor}{^{#2}_{#3}}}
\newcommand{\dindet}[1]{\lceil#1\rceil}
%\newcommand{\dinj}[2]{\texttt{inj}_{#1}({#2})}
\newcommand{\dinL}[2]{\mathsf{inl}_{#1}(#2)}
\newcommand{\dinR}[2]{\mathsf{inr}_{#1}(#2)}
\newcommand{\dcase}[5]{\texttt{case}({#1},{#2}.{#3},{#4}.{#5})}
\newcommand{\dpair}[2]{(#1,#2)}
\newcommand{\dprj}[2]{\mathsf{prj}_{#1}(#2)}

\newcommand{\isType}[2]{#1 \vdash #2~\mathtt{type}}
\newcommand{\elabs}[5]{#1 \vdash #2 \leadsto #3 : #4 \dashv #5}
\newcommand{\expands}[5]{#1; #2 \vdash #3 \leadsto #4 : #5}
\newcommand{\ccexpands}[6]{#1; #2 \vdash_\text{cc} #3 \leadsto #4 : #5 \dashv #6}
\newcommand{\Omegaitem}[2]{#1 \hookrightarrow #2}
\newcommand{\expandAna}[6]{#1 \vdash #2 \Leftarrow #3 \leadsto #4 : #5 \dashv #6}
\newcommand{\expandSyn}[5]{#1 \vdash #2 \Rightarrow #3 \leadsto #4 \dashv #5}
\newcommand{\pexpandAna}[5]{#1 \vdash_{#2} #3 \leadsto #4 \Leftarrow #5}
\newcommand{\pexpandSyn}[5]{#1 \vdash_{#2} #3 \leadsto #4 \Rightarrow #5}
\newcommand{\decodeExp}[2]{#1 \uparrow #2}
\newcommand{\encodeExp}[2]{#1 \downarrow #2}
\newcommand{\hasType}[3]{#1 \vdash #2 : #3}
\newcommand{\hasTypeD}[4]{#1; #2 \vdash #3 : #4}
\newcommand{\isValue}[1]{#1~\mathsf{val}}
\newcommand{\isGround}[1]{#1~\mathsf{ground}}
\newcommand{\isBoxedValue}[1]{#1~\mathsf{boxedval}}
\newcommand{\isIndet}[1]{#1~\mathsf{indet}}
\newcommand{\isFinal}[1]{#1~\mathsf{final}}
\newcommand{\isErr}[2]{#1 \vdash #2~\mathsf{err}}
%% \newcommand{\stepsTo}[2]{#1 \mapsto_{\Delta} #2}
%% TODO first arg should be empty
%% \newcommand{\stepsToD}[3]{#1 \vdash #2 \mapsto #3}
\newcommand{\stepsToD}[3]{#2 \mapsto #3}
\newcommand{\multiStepsTo}[2]{#1 \mapsto^* #2}
\newcommand{\evalsTo}[2]{#1 \Downarrow #2}

%% TODO if feeling obsessive, replace direct uses of \Delta
\newcommand{\hDelta}{\Delta}
\newcommand{\Dunion}[2]{#1 \cup #2}
\newcommand{\idof}[1]{\mathsf{id}(#1)}
\newcommand{\Dbinding}[3]{#1 :: #3[#2]}
\newcommand{\instantiate}[3]{\llbracket#1 / #2\rrbracket #3}
\newcommand{\instantiateB}[2]{\llbracket #1 / #2\rrbracket}

% Contextual dynamics
\newcommand{\evalctx}{\mathcal{E}}
\newcommand{\evalhole}{\circ}
\newcommand{\isevalctx}[1]{#1~\mathsf{evalCtx}}
%% TODO first arg should be empty
%% \newcommand{\reducesE}[3]{#1 \vdash #2 \longrightarrow #3}
\newcommand{\reducesE}[3]{#2 \longrightarrow #3}
\newcommand{\selectEvalCtxR}[2]{#1\{#2\}}
\newcommand{\selectEvalCtx}[3]{#1=\selectEvalCtxR{#2}{#3}}
\newcommand{\maybePremise}[1]{{\textcolor{red}[}#1{\textcolor{red}]}}

\newcommand{\inhole}[2]{\mathsf{inhole}(#1; #2)}

\newcommand{\DoSubst}[3]{[#1/#2]{#3}}

\newcommand{\splat}[1]{\{#1\}_{i < n}}
\newcommand{\taut}[1]{\tau_\text{#1}}
\newcommand{\etxt}[1]{e_\text{#1}}
\newcommand{\dtxt}[1]{d_\text{#1}}
\newcommand{\splices}{\splat{\psi_i}}

\newcommand{\livelitCtxEntry}[4]{\mathsf{livelit}~\livelitname{#1}~\mathsf{at}~#2~\{#3; #4\}}
\newcommand{\expType}{\mathsf{Exp}}
\newcommand{\livelitsIn}[1]{\mathsf{livelits}(#1)}
\newcommand{\protoEnvsOf}[3]{\mathsf{protoenvs}_{#1}(#2; #3)}
\newcommand{\envsOf}[3]{\mathsf{envs}_{#1}(#2; #3)}
\newcommand{\fillof}[2]{\mathsf{fill}_{#1}(#2)}
\newcommand{\resumeof}[1]{\mathsf{resume}(#1)}
