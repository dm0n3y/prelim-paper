\begin{figure}
  \vspace{-3px}
  \[
  \arraycolsep=3pt\begin{array}{rlrl}
    \text{tile sequence} & \tiles^s & ::= & \tile^s_1\dots\tile^s_n \\
    \text{pattern tile} & \tile^{\pat} & ::= &
      \ophole ~\vert~
      \svar{x} \\
    & & \vert &
      % \sann{}{\tiles^{\typ}} ~\vert~
      \binhole ~\vert~
      \sprod{}{} \\
    & & \vert &
      \sparen{\tiles^{\pat}} \\
    \text{expression tile} & \tile^{\expr} & ::= &
      \ophole ~\vert~
      % \sboollit{b} ~\vert~
      \snumlit{n} ~\vert~
      \svar{x} \\
    & & \vert &
      \sparen{\tiles^{\expr}} \\
    & & \vert &
      \binhole ~\vert~
      % \sap{}{} ~\vert~
      \sprod{}{} ~\vert~
      \splus{}{} ~\vert~
      \smult{}{} \\
      % \sequals{}{} ~\vert~
    & & \vert &
      \slam{\tiles^{\pat}}{} \\
    & & \vert &
      \slet{\tiles^{\pat}}{\tiles^{\expr}}{}\\\\ % ~\vert~
      % \scond{}{\tiles^{\expr}}{}
  \end{array}\]
  \caption{
    Syntax of pattern and expression tiles.
    \note{add operator holes, add shapes, show shapes but note informal,
    maybe consider adding a fake postfix op}
  }
  \label{fig:tile-syntax}
\end{figure}
