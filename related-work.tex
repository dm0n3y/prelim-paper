\section{Related Work}\label{sec:related-work}

\paragraph{Structure editing}
Structure editing has a long history, dating back
to the introduction of the Cornell Program Synthesizer
\cite{Cornell} in 1981.
\note{add words}

The most popular structure editors today
are block-based editors such as Scratch \cite{scratch}, Snap!,
App Inventor, Alice, and many more.
In these editors, the user authors a program
like the one shown to the right
by dragging-and-dropping blocks together on a canvas.
Each block corresponds to a syntactic form of
the underlying language and is shaped, based on its
sort, to visually indicate how it should be placed relative
to other blocks.
\tylr~ employs a similar metaphor of syntactic-forms-as-puzzle-pieces,
but uses a uniform shape system across all sorts,
thereby simplifying potential language extensions.

Block-based editors have seen great success in recent years
at introducing children to programming, but they soon
become unwieldy once users start creating and maintaining
larger or more expression-oriented programs.
For example---adapting an observation by
Brown \etal \cite{no-keyboard-cripples}---constructing
the small calculation shown in
Figure ?? involves assembling seven blocks, each
requiring a sequence of gestures to find and
drag-and-drop the appropriate block into the right spot
in the canvas;
the equivalent construction in a text editor or \tylr~ would
take seven keypresses, summing to much less effort.
\note{need to make linear construction theme more explicit}

Meanwhile, in a user study of block-based editing
involving large refactoring tasks,
Holwerda and Hermans elicited post-task user responses
on the cognitive dimensions of block-based editing and
found that \emph{viscosity} was the most commented-on
dimension with 24 remarks.
Half (12) were positive, a majority of which
were about the ease of refactoring when the selected
elements corresponded to complete syntactic terms.
Of the negative half, half (6) were about the difficulty
of refactoring when the desired selection
does not correspond to a complete term;
for example, in Figure ??, dragging the \code{move} block out of
the \code{repeat} block drags all the statements below
along with it, thereby requiring multiple gestures
in sum to select and remove the single \code{move} block.

The issue of inexpressive selection is not particular to
block-based editors.
Any structure editor that directly uses the
concrete or abstract language syntax to model
its edit state fundamentally limits selections to complete
language terms.
\note{add words + figure comparing text, structure, and tile
editing model}


Recognizing the pitfalls of strictly term-based
editing, some structure editors
employ hybrid editing models, using structural editing
for large syntactic forms while deferring to text
editing at the leaves.
\note{talk about how this approach eases linear construction
of operator sequences, supports arbitrary selections within
a text leaf, but loses the benefits of structure editing
at those levels, moreover cannot extend arbitrary selections
beyond a given text leaf}

\note{talk about MPS being a strict structure editor
with extensive additional design and engineering to
support certain text-like editing capabilities, \eg,
linear construction of opseqs (using side transforms),
dangling parentheses AST annotations to support
parenthesizing arbitrary ranges}

Text-like keyboard-driven structure editors
like MPS and \tylr~ carry additional usability challenges
compared to block-based editors due to the lack of an
open canvas...
\note{talk about how deletion is forced to be destructive,
MPS mitigates by saving first child if same sort,
otherwise removes everything }

\note{paragraph on structural editing plug-ins for text editors}

Hazelnut \cite{Hazelnut} is a structure editor calculus
similar in spirit to \ty, but with the aim of
preserving well-typedness of its edit states.
Hazelnut achieves this aim by developing a typed
expression language featuring non-empty holes, which
are automatically inserted by its type-directed action
semantics to cordon off ill-typed portions of the program
as they arise during editing.
Editing in Hazelnut is strictly term-based, however, posing
the same usability issues described earlier.
In contrast, \ty~ and \tylr~ are untyped, the focus in
this work being on developing more ergonomic interactions
for maintaining syntactic well-formedness.

\paragraph{Nested words}
\ty~ segments are closely related to \emph{nested words}
\cite{nested-words},
a data representation that encodes both linear
ordering and hierarchically nested matching
of items.
A nested word consists of a sequence of letters,
some pairs of which are matching and organized
in a well nested fashion, \ie, pairs do not cross.
Segments encode a similar matching between the
constituent shards of tiles contained therein,
though \ty~ permits $n$-ary matching as opposed
to the strictly binary matching of nested words.
Nested words are well-studied from the lens of
formal language theory, but we are not aware of
any prior work that has studied editing of nested words
\ala restructuring in \tylr.

\paragraph{Zippers}
First documented by Huet
\cite{zipper}, the zipper pattern has been used to model a range of systems,
including text editors \cite{lazy-functional-incremental-parsing},
file systems \cite{zipper-fs}, and window managers \cite{window-manager}.
If the underlying structure is hierarchical,
zippers typically ask the client to navigate
in a structure-enforcing two-dimensional manner: left and right to traverse
sibling nodes, up and down to traverse parent-child relations.
In contrast, \ty~ supports a one-dimensional interaction model
in the presence of hierarchy, using disassembly to flatten
hierarchical structures as needed to support linear traversal
and partial selection.
We note that this model is distinct from using an ordered tree
traversal to linearize movement, which is still involves
strictly hierarchical navigation rather than
left-to-right navigation through a serialization of the edit state.

Some work on zippers focuses on the mechanistic \cite{derivative-zippers,clowns-jokers}
and automatic \cite{syz} generation of zipper forms
from datatypes;
these techniques are orthogonal to our tile-based zipper design
and may prove useful in future work on
automatically generating tile-based zippers
from a given syntax.

\note{mention Parsing With Zippers}

% \paragraph{Incremental parsing}
% Zipper operations in \ty~ rely on incremental and partial
% parsing to reassemble shards as the opportunity arises.
% Existing incremental parsing techniques

% \begin{itemize}
%   \item structure editing
%   \begin{itemize}
%     \item Design Requirements for More Flexible Structured Editors from a Study of Programmers' Text Editing
%       by Amy Ko \& Brad Myers
%     \begin{itemize}
%         \item list edits (23\%):
%           removing a list element and its delimiter (8\%);
%           moving the right list delimiter(3\%);
%           ``flattening'' a list inside of a list (2\%)
%         \item infix expressions (15\%):
%             replacing the infix expression with its left (8\%) or right (1\%) operand;
%             ``wrapping'' a left or right operand (4\%);
%             unwrapping a left or right operand (2\%)
%         \item prefix expressions (1\%):
%             applying a prefix operator to an expression (81\%);
%             ``unwrapping'' a prefix operand (11\%)
%         \item that amounts to 6\% of edits clearly involving selecting and
%           moving/removing non-term selections, or 10\% of edits not involving
%           modifying names
%     \end{itemize}
%   \end{itemize}
% \end{itemize}