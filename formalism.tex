% !TEX root = prelim-paper.tex

\section{Editor Calculus}\label{sec:formalism}

\note{think about name of calculus as opposed to implementation}

\subsection{Tiles, tokens, \& selections}
\begin{figure}
  \vspace{-3px}
  \[
  \arraycolsep=3pt\begin{array}{rlrl}
      \mathsf{PatToken} & \shard^{\pat} & ::= &
        \shardlit{\ophole} ~\vert~
        \shardlit{\svar{x}} ~\vert~
        \shardlit{\binhole} ~\vert~
        \shardlit{\sprod{}{}} ~\vert~
        \shardlit{\texttt{(}} ~\vert~
        \shardlit{\texttt{)}} \\
      \mathsf{ExpToken} & \shard^{\expr} & ::= &
        \shardlit{\ophole} ~\vert~
        \shardlit{\sboollit{b}} ~\vert~
        \shardlit{\snumlit{n}} ~\vert~
        \shardlit{\svar{x}} ~\vert~
        \shardlit{\binhole} ~\vert~
        \shardlit{\sprod{}{}} ~\vert~
        \shardlit{\splus{}{}} ~\vert~
        \shardlit{\smult{}{}} \\
      & & \vert &
        \shardlit{\texttt{(}} ~\vert~
        \shardlit{\texttt{)}} ~\vert~
        \shardlit{$\lambda$} ~\vert~
        \shardlit{\texttt{.}} ~\vert~
        \shardlit{\texttt{let}} ~\vert~
        \shardlit{\texttt{=}} ~\vert~
        \shardlit{\texttt{in}}
  \end{array}\]
  \caption{
    Syntax of pattern and expression tokens.
  }
  \label{fig:token-syntax}
\end{figure}
\begin{figure}
  \vspace{-3px}
  \[
  \arraycolsep=3pt\begin{array}{rlrl}
      \mathsf{Selection} & \selection & ::= &
        \selem_1\dots\selem_n \\
      \mathsf{SelectedElement} & \selem & ::= &
        \tile^{\pat} ~\vert~
        \tile^{\expr} ~\vert~
        \token^{\pat} ~\vert~
        \token^{\expr}
  \end{array}\]
  \caption{
    Syntax of selections.
  }
  \label{fig:selection-syntax}
\end{figure}
\begin{figure}
  \newcommand{\dd}{\searrow}
  \judgbox{\disassemblesDown{\selem}{\selection}}{$\selem$ disassembles down to $\selection$}
  \[
    \arraycolsep=3pt
    \begin{array}{rcl}
      \sparen{\tiles^s} & \dd & \shardlit{\texttt{(}}~\tiles^s~\shardlit{\texttt{)}} \\
      \slam{\tiles^{\pat}} & \dd & \shardlit{\lambda}~\tiles^{\pat}~\shardlit{\texttt{.}} \\
      \slet{\tiles^{\pat}}{\tiles^{\expr}}{} & \dd & \shardlit{\texttt{let}}~\tiles^{\pat}~\shardlit{\texttt{=}}~\tiles^{\expr}~\shardlit{\texttt{in}}
  \end{array}\]
  \caption{
    Disassembling a selected element $\selem$.
    If $\selem$ has a syntactic form not listed, then we assume that $\disassemblesDown{\selem}{\cdot}$.
  }
  \label{fig:disassemble-tile}
\end{figure}
\begin{figure}
\judgbox{\stepDisassembleSelection{\selection_1}{\selection_2}}{$\selection_1$ disassembles to $\selection_2$}
\begin{mathpar}
  \inferrule[]{
    \disassembleTile{\tau^s} = \Psi'
  }{
    \stepDisassembleSelection{\tau^s\Psi}{\Psi'\Psi}
  }\hspace{40pt}
  \inferrule[]{
    \stepDisassembleSelection{\Psi}{\Psi'}
  }{
    \stepDisassembleSelection{\psi\Psi}{\psi\Psi'}
  }
\end{mathpar}
\caption{
  Disassembling a selection
}
\label{fig:disassemble-selection}
\end{figure}

\note{
  discuss the syntax of
  tiles (Fig \ref{fig:tile-syntax}),
  tokens (Fig \ref{fig:token-syntax}),
  selections (Fig \ref{fig:selection-syntax})
}

\note{
  discuss
    disassembly of tiles (Fig \ref{fig:disassemble-tile}),
    disassembly of selections (Fig \ref{fig:disassemble-selection}),
    $\succeq$ partial order and parsing function
}

\begin{definition}
  $\searrow^*$ is the reflexive transitive
  closure of $\searrow$.
\end{definition}

\begin{lemma}
  $\searrow^*$ is a partial order.
\end{lemma}

\begin{lemma}
  For every selection $\Psi$, there is a unique maximal
  element $\Psi'$ such that $\Psi'\searrow^*\Psi$.
\end{lemma}

\begin{definition}
  \note{define function that returns unique maximal element of selection}
\end{definition}


\subsection{Frames \& zippers}
\begin{figure}
  \vspace{-3px}
  \[
  \arraycolsep=3pt\begin{array}{rlrl}
      \mathsf{TilesFrame} & \tframe^s & ::= & \tframelit{\tiles^s\_\tiles^s}{\tfrelem^s} \\
      % \mathsf{Tile}^{\typ} & \tile^{\typ} & ::= &
      %     % \tnum ~\vert~
      %     \shole ~\vert~
      %     \sbool ~\vert~
      %     \snum ~\vert~
      %     \sarr{}{} ~\vert~
      %     \sprod{}{} ~\vert~
      %     \sparen{\tiles^{\typ}}\\
      \mathsf{PatTileFrame} & \tfrelem^{\pat} & ::= &
        \tframelit{\sparen{\_}}{\tframe^{\pat}} \\
      & & \vert &
        \tframelit{\slam{\_}{}}{\tframe^{\expr}} \\
      & & \vert &
        \tframelit{\slet{\_}{\tiles^{\expr}}{}}{\tframe^{\expr}} \\
      \mathsf{ExpTileFrame} & \tfrelem^{\expr} & ::= &
        \tframelit{\sparen{\_}}{\tframe^{\expr}} \\
      & & \vert &
        \tframelit{\slet{\tiles^{\pat}}{\_}{}}{\tframe^{\expr}}
        % \scond{}{\tiles^{\expr}}{}
  \end{array}\]
  \caption{
    Syntax of pattern and expression frames.
  }
  \label{fig:tile-syntax}
\end{figure}

\begin{figure}
  \[\arraycolsep=3pt\begin{array}{rlrl}
      \text{zipper} & \zip & ::= & \zipper{\subject}{\zframe} \\
      \text{subject} & \subject & ::= &
        \pointing{\selection}{\selection} ~\vert~
        \selecting{\selection}{\selection}{\selection} ~\vert~
        \restructuring{\selection}{\selection}{\selection} \\
      \text{frame} & \zframe & ::= &
        \tfrelem^{\pat} ~\vert~ \tfrelem^{\expr}
 \end{array}\]
  \caption{
    Zipper syntax
  }
  \label{fig:zipper-syntax}
\end{figure}

\begin{figure}
  \vspace{-3px}
  \[
  \setlength{\fboxsep}{1pt}
  \arraycolsep=3pt\def\arraystretch{1.4}\begin{array}{rcl}
      \disassembleTileFrame{
        \tframelit{
          \sparen{\_}
        }{
          \tframelit{\tiles^s_1\_\tiles^s_2}{\tfrelem^{\pat}}
        }
      } & = &
        \zipper{
          \pointing{\tiles^s_1\tokenLit{\texttt{(}}~}{~\tokenLit{\texttt{)}}\tiles^s_2}
        }{
          \tfrelem^{\pat}
        } \\

      \disassembleTileFrame{
        \tframelit{
          \slam{\_}{}
        }{
          \tframelit{\tiles^{\expr}_1\_\tiles^{\expr}_2}{\tfrelem^{\expr}}
        }
      } & = &
        \zipper{
          \pointing{\tiles^{\expr}_1\tokenLit{\lambda}~}{~\tokenLit{\texttt{.}}\tiles^{\expr}_2}
        }{
          \tfrelem^{\expr}
        } \\

      \disassembleTileFrame{
        \tframelit{
          \slet{\_}{\tiles^{\expr}_0}{}
        }{
          \tframelit{\tiles^{\expr}_1\_\tiles^{\expr}_2}{\tfrelem^{\expr}}
        }
      } & = & \\
        \zipper{
          \pointing{
            \tiles^{\expr}_1\tokenLit{\texttt{let}}~
          }{
            ~\tokenLit{\texttt{=}}~\tiles^{\expr}_0\tokenLit{\texttt{in}}~\tiles^{\expr}_2
          }
        }{
          \tfrelem^{\expr}
        } \\

      \disassembleTileFrame{
        \tframelit{
          \slet{\tiles^{\pat}}{\_}{}
        }{
          \tframelit{\tiles^{\expr}_1\_\tiles^{\expr}_2}{\tfrelem^{\expr}}
        }
      } & = & \\
        \zipper{
          \pointing{
            \tiles^{\expr}_1\tokenLit{\texttt{let}}~\tiles^{\pat}~\tokenLit{\texttt{=}}~
          }{
            ~\tokenLit{\texttt{in}}~\tiles^{\expr}_2
          }
        }{
          \tfrelem^{\expr}
        }
  \end{array}\]
  \caption{
    Disassembling a tile frame.
  }
  \label{fig:disassemble-tile}
\end{figure}

\begin{definition}
  $\zeta^s_1\nearrow\zeta^s_2$ if $\disassembleTileFrame{\zeta^s_1} = \zeta^s_2$.
  \note{defined this for symmetry with diassembling of tiles}
\end{definition}

\begin{definition}
  $\nearrow^*$ is the reflexive transitive closure of $\nearrow$.
\end{definition}

\begin{lemma}
  $\nearrow^*$ is a partial order.
\end{lemma}

\begin{lemma}
  For every zipper $\zip^s_1 = \zipper{\pointing{\selection_1}{\selection_2}}{\tfrelem^s}$, there exists
  a unique minimal element $\zip^s_2$ such that $\zip^s_2 \nearrow^* \zip^s_1$.
\end{lemma}

\begin{figure}
  \vspace{-3px}
  \[
  \arraycolsep=3pt\begin{array}{rlrl}
      \mathsf{Action} & \action & ::= &
        \actionlit{mark} ~\vert~
        \actionlit{move}~\direction ~\vert~
        \actionlit{delete} ~\vert~
        \actionlit{construct}~\tile^s \\
      \mathsf{Direction} & \direction & ::= &
        \texttt{left} ~\vert~
        \texttt{right}
  \end{array}\]
  \caption{Syntax of actions}
  \label{fig:action-syntax}
\end{figure}

\begin{figure*}
  \judgbox{\performAction{\editState_1}{\action}{\editState_2}}{$\editState_1$ transitions via $\alpha$ to $\editState_2$}
  \begin{mathpar}
  \inferrule[MoveRightPast]{
    \disassemblesDown{\selem}{\cdot}
  }{
    \performAction{
      \zipper{\pointing{\selection_1}{\selem\selection_2}}{\tfrelem^s}
    }{
      \actionlit{move}\ \texttt{right}
    }{
      \zipper{\pointing{\selection_1\selem}{\selection_2}}{\tfrelem^s}
    }
  } \\
  \inferrule[MoveRightEnter]{
    \disassemblesDown{\selem}{\selem'\selection_3}
  }{
    \performAction{
      \zipper{\pointing{\selection_1}{\selem\selection_2}}{\tfrelem^s}
    }{
      \actionlit{move}\ \texttt{right}
    }{
      \parseZipper{\zipper{\pointing{\selection_1\selem'}{\selection_3\selection_2}}{\tfrelem^s}}
    }
  } \ \ \ \ \ \
  \inferrule[MoveRightExitTile]{
    \disassemblesUp{
      \tfrelem^s
    }{
      \zipper{\pointing{\selection_2}{\selem\selection_3}}{\tfrelem^{s'}}
    }
  }{
    \performAction{
      \zipper{\pointing{\selection_1}{\cdot}}{\tfrelem^s}
    }{
      \actionlit{move}\ \texttt{right}
    }{
      \parseZipper{
        \zipper{\pointing{\selection_2\selection_1\selem}{\selection_3}}{\tfrelem^{s'}}
      }
    }
  } \\
  \inferrule[ConstructTile]{
    \disassemblesDown{\tile^s}{\token^s\selection_3} \\
    \fixHolesSelections{\rtipconc{s}}{\selection_1\token^s}{\selection_3\selection_2}{\ltipconc{s}}{\selection_4}{\selection_5}
  }{
    \performAction{
      \zipper{\pointing{\selection_1}{\selection_2}}{\tfrelem^s}
    }{
      \actionlit{construct}\ \tile^s
    }{
      \parseZipper{
        \zipper{\pointing{\selection_4}{\selection_5}}{\tfrelem^{s}}
      }
    }
  } \ \ \ \ \ \ \ \
  \inferrule[MarkPointing]{
  }{
    \performAction{
      \zipper{\pointing{\selection_1}{\selection_2}}{\tfrelem^s}
    }{
      \actionlit{mark}
    }{
      \zipper{\selecting{\selection_1}{\strut~\cdot~}{\selection_2}}{\tfrelem^s}
    }
  } \\
  \inferrule[SelectLeftAtom]{
    \disassemblesDown{\selem}{\cdot} \\
    \parseSelection{\selem\selection_2} = \selection'_2
  }{
    \performAction{
      \zipper{\selecting{\selection_1\selem}{\selection_2}{\selection_3}}{\tfrelem^s}
    }{
      \actionlit{move}\ \texttt{left}
    }{
      \zipper{\selecting{\selection_1}{\selection'_2}{\selection_3}}{\tfrelem^s}
    }
  } \ \ \
  \inferrule[SelectLeftDisassembles]{
    \disassemblesDown{\selem}{\selection_4\selem'} \\
    \parseSelection{\selem'\selection_2} = \selection'_2
  }{
    \performAction{
      \zipper{\selecting{\selection_1\selem}{\selection_2}{\selection_3}}{\tfrelem^s}
    }{
      \actionlit{move}\ \texttt{left}
    }{
      \zipper{\selecting{\selection_1\selection_4}{\selection'_2}{\selection_3}}{\tfrelem^s}
    }
  } \ \ \
  \inferrule[SelectLeftExit]{
    \disassemblesUp{
      \tfrelem^s
    }{
      \zipper{\pointing{\selection_3\selem}{\selection_4}}{\tfrelem^{s'}}
    }
  }{
    \performAction{
      \zipper{\selecting{\cdot}{\selection_1}{\selection_2}}{\tfrelem^s}
    }{
      \actionlit{move}\ \texttt{left}
    }{
      \zipper{\selecting{\selection_3}{\selection_1\selem}{\selection_2\selection_4}}{\tfrelem^{s'}}
    }
  } \\
  \inferrule[SelectRightAtom]{
    \disassemblesDown{\selem}{\cdot} \\
    \parseZipper{\zipper{\pointing{\selection_1\selem}{\selection_3}}{\tfrelem^s}} =
      \zipper{\pointing{\selection'_1}{\selection'_3}}{\tfrelem^{s'}}
  }{
    \performAction{
      \zipper{\selecting{\selection_1}{\selem\selection_2}{\selection_3}}{\tfrelem^s}
    }{
      \actionlit{move}\ \texttt{right}
    }{
      \zipper{\selecting{\selection'_1}{\selection_2}{\selection'_3}}{\tfrelem^{s'}}
    }
  } \ \ \ \ \ \ \ \
  \inferrule[SelectRightDisassembles]{
    \disassemblesDown{\selem}{\selem'\selection_4}
  }{
    \performAction{
      \zipper{\selecting{\selection_1}{\selem\selection_2}{\selection_3}}{\tfrelem^s}
    }{
      \actionlit{move}\ \texttt{right}
    }{
      \zipper{\selecting{\selection_1\selem'}{\selection_4\selection_2}{\selection_3}}{\tfrelem^s}
    }
  } \\
  \inferrule[MarkSelecting]{
    \matchesSort{\leftTip{\selection_2}}{\rightTip{\selection_2}} \\
    \fixHolesSelections{\rtipconc{s}}{\selection_1}{\selection_3}{\ltipconc{s}}{\selection'_1}{\selection'_3}
  }{
    \performAction{
      \zipper{\selecting{\selection_1}{\selection_2}{\selection_3}}{\tfrelem^s}
    }{
      \actionlit{mark}
    }{
      \zipper{\restructuring{\selection'_1}{\selection_2}{\selection'_3}}{\tfrelem^s}
    }
  } \\
  \inferrule[MoveRightRestructuringNotWhole]{
    \text{\note{not\ }}\wholeSelection{s'}{\selection_2}
  }{
    \performAction{
      \zipper{\restructuring{\selection_1}{\selection_2}{\tile^{s'}\selection_3}}{\tfrelem^s}
    }{
      \actionlit{move}\ \texttt{right}
    }{
      \zipper{\restructuring{\selection_1\tile^{s'}}{\selection_2}{\selection_3}}{\tfrelem^s}
    }
  } \\
  \inferrule[MoveRightRestructuringWhole]{
    \wholeSelection{s}{\selection_2} \\
    \performAction{
      \zipper{\pointing{\selection_1}{\selection_3}}{\tfrelem^s_1}
    }{
      \actionlit{move}\ \texttt{right}
    }{
      \zipper{\pointing{\selection_4}{\selection_5}}{\tfrelem^{s'}_2}
    }
  }{
    \performAction{
      \zipper{\restructuring{\selection_1}{\selection_2}{\selection_3}}{\tfrelem^s_1}
    }{
      \actionlit{move}\ \texttt{right}
    }{
      \zipper{\restructuring{\selection_4}{\selection_2}{\selection_5}}{\tfrelem^{s'}_2}
    }
  } \\
  \inferrule[Delete]{
    \fixHolesSelections{\rtipconc{s}}{\filterTiles{s}{\selection_1}}{\filterTiles{s}{\selection_3}}{\ltipconc{s}}{\selection'_1}{\selection'_3}
  }{
    \performAction{
      \zipper{\restructuring{\selection_1}{\selection_2}{\selection_3}}{\tfrelem^s}
    }{
      \actionlit{delete}
    }{
      \zipper{\pointing{\selection'_1}{\selection'_3}}{\tfrelem^s}
    }
  } \\
  \inferrule[MarkRestructuringNotWhole]{
    \text{\note{not\ }}\wholeSelection{s'}{\selection_2} \\
    \fixHolesSelections{\rtipconc{s}}{\selection_1}{\selection_2\selection_3}{\ltipconc{s}}{\selection_4}{\selection_5}
  }{
    \performAction{
      \zipper{\restructuring{\selection_1}{\selection_2}{\selection_3}}{\tfrelem^s}
    }{
      \actionlit{mark}
    }{
      \parseZipper{\zipper{\pointing{\selection_4}{\selection_5}}{\tfrelem^s}}
    }
  } \\
  \inferrule[MarkRestructuringWhole]{
    \wholeSelection{s}{\selection_2} \\
    \fixHolesSelections{\rtipconc{s}}{\selection_1}{\selection_2\selection_3}{\ltipconc{s}}{\selection_4}{\selection_5}
  }{
    \performAction{
      \zipper{\restructuring{\selection_1}{\selection_2}{\selection_3}}{\tfrelem^s}
    }{
      \actionlit{mark}
    }{
      \parseZipper{\zipper{\pointing{\selection_4}{\selection_5}}{\tfrelem^s}}
    }
  }
  \end{mathpar}

  \vspace{-2px}
  \CaptionLabel{Action performing \note{prevent empty hole tile construction, add justification for constructing tiles rather than tokens}}{fig:perform-action}
  \vspace{-2px}
  \end{figure*}

\note{give an example of a frame to clarify what's going on}

\subsubsection{Actions}
\note{
  Define actions more generally than \tylr~ implementation,
  such that delimiters are assigned sorts to their ends,
  selections are not restricted to shards and
  same-sort tiles, and only selections whose ends are the
  same may enter restructuring mode.
}

% (2 * [3])  ->  (2 * (3 + [_]))

\begin{itemize}
  \item explain action judgment + choice rules
  \item theorems
  \begin{itemize}
    \item movability
    \item selectability
    \item restructuring is sound and complete
  \end{itemize}
\end{itemize}

\subsection{Typechecking}
\note{
  Define tree-structured variants of tiles and frames.
  Define precedence parser that converts linear forms to tree forms
  (or assume existence of such parser?).
  Define type
}
