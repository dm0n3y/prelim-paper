% !TEX root = prelim-paper.tex

\section{Editor Calculus}\label{sec:formalism}

\note{think about name of calculus as opposed to implementation}

\subsection{Tiles, tokens, \& selections}
\begin{figure}
  \vspace{-3px}
  \[
  \arraycolsep=3pt\begin{array}{rlrl}
      \mathsf{PatToken} & \shard^{\pat} & ::= &
        \shardlit{\ophole} ~\vert~
        \shardlit{\svar{x}} ~\vert~
        \shardlit{\binhole} ~\vert~
        \shardlit{\sprod{}{}} ~\vert~
        \shardlit{\texttt{(}} ~\vert~
        \shardlit{\texttt{)}} \\
      \mathsf{ExpToken} & \shard^{\expr} & ::= &
        \shardlit{\ophole} ~\vert~
        \shardlit{\sboollit{b}} ~\vert~
        \shardlit{\snumlit{n}} ~\vert~
        \shardlit{\svar{x}} ~\vert~
        \shardlit{\binhole} ~\vert~
        \shardlit{\sprod{}{}} ~\vert~
        \shardlit{\splus{}{}} ~\vert~
        \shardlit{\smult{}{}} \\
      & & \vert &
        \shardlit{\texttt{(}} ~\vert~
        \shardlit{\texttt{)}} ~\vert~
        \shardlit{$\lambda$} ~\vert~
        \shardlit{\texttt{.}} ~\vert~
        \shardlit{\texttt{let}} ~\vert~
        \shardlit{\texttt{=}} ~\vert~
        \shardlit{\texttt{in}}
  \end{array}\]
  \caption{
    Syntax of pattern and expression tokens.
  }
  \label{fig:token-syntax}
\end{figure}
\begin{figure}
  \vspace{-3px}
  \[
  \arraycolsep=3pt\begin{array}{rlrl}
      \mathsf{Selection} & \selection & ::= &
        \selem_1\dots\selem_n \\
      \mathsf{SelectedElement} & \selem & ::= &
        \tile^{\pat} ~\vert~
        \tile^{\expr} ~\vert~
        \token^{\pat} ~\vert~
        \token^{\expr}
  \end{array}\]
  \caption{
    Syntax of selections.
  }
  \label{fig:selection-syntax}
\end{figure}
\begin{figure}
  \newcommand{\dd}{\searrow}
  \judgbox{\disassemblesDown{\selem}{\selection}}{$\selem$ disassembles down to $\selection$}
  \[
    \arraycolsep=3pt
    \begin{array}{rcl}
      \sparen{\tiles^s} & \dd & \shardlit{\texttt{(}}~\tiles^s~\shardlit{\texttt{)}} \\
      \slam{\tiles^{\pat}} & \dd & \shardlit{\lambda}~\tiles^{\pat}~\shardlit{\texttt{.}} \\
      \slet{\tiles^{\pat}}{\tiles^{\expr}}{} & \dd & \shardlit{\texttt{let}}~\tiles^{\pat}~\shardlit{\texttt{=}}~\tiles^{\expr}~\shardlit{\texttt{in}}
  \end{array}\]
  \caption{
    Disassembling a selected element $\selem$.
    If $\selem$ has a syntactic form not listed, then we assume that $\disassemblesDown{\selem}{\cdot}$.
  }
  \label{fig:disassemble-tile}
\end{figure}
\begin{figure}
\judgbox{\stepDisassembleSelection{\selection_1}{\selection_2}}{$\selection_1$ disassembles to $\selection_2$}
\begin{mathpar}
  \inferrule[]{
    \disassembleTile{\tau^s} = \Psi'
  }{
    \stepDisassembleSelection{\tau^s\Psi}{\Psi'\Psi}
  }\hspace{40pt}
  \inferrule[]{
    \stepDisassembleSelection{\Psi}{\Psi'}
  }{
    \stepDisassembleSelection{\psi\Psi}{\psi\Psi'}
  }
\end{mathpar}
\caption{
  Disassembling a selection
}
\label{fig:disassemble-selection}
\end{figure}

\note{
  discuss the syntax of
  tiles (Fig \ref{fig:tile-syntax}),
  tokens (Fig \ref{fig:token-syntax}),
  selections (Fig \ref{fig:selection-syntax})
}

\note{
  discuss
    disassembly of tiles (Fig \ref{fig:disassemble-tile}),
    disassembly of selections (Fig \ref{fig:disassemble-selection}),
    $\succeq$ partial order and parsing function
}

\begin{definition}
  $\disassembleSelection{\Psi}{\Psi'}$ if $\Psi = \Psi'$ or
  $\Psi \searrow^* \Psi'$, where $\searrow^*$ is the transitive
  closure of $\searrow$.
\end{definition}

\begin{lemma}
  $\disassembleSelection{}{}$ is a partial order.
\end{lemma}

\begin{lemma}
  For every selection $\Psi$, there is a unique maximal
  element $\Psi'$ such that $\disassembleSelection{\Psi'}{\Psi}$.
\end{lemma}

\begin{definition}
  \note{define function that returns unique maximal element of selection}
\end{definition}


\subsection{Frames \& zippers}
\begin{figure}
  \vspace{-3px}
  \[
  \arraycolsep=3pt\begin{array}{rlrl}
      \mathsf{TilesFrame} & \tframe^s & ::= & \tframelit{\tiles^s\_\tiles^s}{\tfrelem^s} \\
      % \mathsf{Tile}^{\typ} & \tile^{\typ} & ::= &
      %     % \tnum ~\vert~
      %     \shole ~\vert~
      %     \sbool ~\vert~
      %     \snum ~\vert~
      %     \sarr{}{} ~\vert~
      %     \sprod{}{} ~\vert~
      %     \sparen{\tiles^{\typ}}\\
      \mathsf{PatTileFrame} & \tfrelem^{\pat} & ::= &
        \tframelit{\sparen{\_}}{\tframe^{\pat}} \\
      & & \vert &
        \tframelit{\slam{\_}{}}{\tframe^{\expr}} \\
      & & \vert &
        \tframelit{\slet{\_}{\tiles^{\expr}}{}}{\tframe^{\expr}} \\
      \mathsf{ExpTileFrame} & \tfrelem^{\expr} & ::= &
        \tframelit{\sparen{\_}}{\tframe^{\expr}} \\
      & & \vert &
        \tframelit{\slet{\tiles^{\pat}}{\_}{}}{\tframe^{\expr}}
        % \scond{}{\tiles^{\expr}}{}
  \end{array}\]
  \caption{
    Syntax of pattern and expression frames.
  }
  \label{fig:tile-syntax}
\end{figure}

\begin{figure}
  \[\arraycolsep=3pt\begin{array}{rlrl}
      \text{zipper} & \zip & ::= & \zipper{\subject}{\zframe} \\
      \text{subject} & \subject & ::= &
        \pointing{\selection}{\selection} ~\vert~
        \selecting{\selection}{\selection}{\selection} ~\vert~
        \restructuring{\selection}{\selection}{\selection} \\
      \text{frame} & \zframe & ::= &
        \tfrelem^{\pat} ~\vert~ \tfrelem^{\expr}
 \end{array}\]
  \caption{
    Zipper syntax
  }
  \label{fig:zipper-syntax}
\end{figure}


\subsubsection{Actions}
\note{
  Define actions more generally than \tylr~ implementation,
  such that delimiters are assigned sorts to their ends,
  selections are not restricted to shards and
  same-sort tiles, and only selections whose ends are the
  same may enter restructuring mode.
}

% (2 * [3])  ->  (2 * (3 + [_]))

\begin{itemize}
  \item explain action judgment + choice rules
  \item theorems
  \begin{itemize}
    \item movability
    \item selectability
    \item restructuring is sound and complete
  \end{itemize}
\end{itemize}

\subsection{Typechecking}
\note{
  Define tree-structured variants of tiles and frames.
  Define precedence parser that converts linear forms to tree forms
  (or assume existence of such parser?).
  Define type
}
