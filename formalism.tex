% !TEX root = prelim-paper.tex

\section{Tile-based Editor Calculus}\label{sec:formalism}

We now precisely characterize tile-based editing
as a calculus called \ty.
The punchline of \ty~ (Section \ref{sec:actions})
is the action judgment
$\performAction{\editState_1}{\action}{\editState_2}$,
which sends edit state $\editState_1$ to $\editState_2$
via action $\action$; and its governing invariants,
which state that every action-reachable edit state
can be parsed into a well-formed program term.

Edit states $Z$ follow a variant of the well-known \emph{zipper}
pattern \cite{zipper}, which decomposes a structure of interest
into a focused substructure $B$ and its surrounding context $M$.
In this work, we refer to $B$ as the \emph{subject}
and $M$ as the \emph{frame}.
\[
\arraycolsep=4pt
\begin{array}{rlrl}
  \text{zipper} & \zip & ::= & \zipper{\subject}{\zframe}
\end{array}
\]
Our design diverges from prior art in our pursuit
of linear text-like interactivity.
Prior zipper designs are strictly hierarchical and ask
the client to navigate in a structure-aware two-dimensional
fashion: left and right to traverse siblings,
up and down to traverse parent-child relations.
In contrast, we aim to support a one-dimensional interaction
model:
it should be possible to navigate freely by moving solely
left and right \note{might need to clarify or change wording
here to address comparison with ordered traversals of trees};
furthermore it should be possible to specify
arbitrary subranges of a one-dimensional serialization
of the total structure.
Toward these ends, actions in \ty~ rely
on a form of incremental unparsing and reparsing by which
hierarchical structures may be ``flattened'' as needed
to support linear traversal and partial selection.

Our formal presentation builds upon the term and tile syntax
described earlier in Figure \ref{fig:term-tile-syntax} of
Section \ref{sec:overview}.
We will begin by introducing shards
and describing their relation to tiles via disassembly
and reassembly (Section ??) .
We will then describe how tiles and shards relate to
terms (Section ??). In particular, we will show that
we can successfully parse a sequence of tiles
into a term provided that they ``fit'' together in
certain ways.

As a result, we need not limit ourselves to
the highly structured terms
but can instead organize zippers around the
more flexible tiles and shards (Section ??).
This paves the way for linear navigation and
flexible selection when we define actions operating
on zippers (Section ??).
Meanwhile, by taking care to
maintain proper ``fit'' and tile assembly along the
way, actions ensure that reachable edit states can
be successfully parsed into a program term.

% We will begin in Section \ref{sec:terms-tiles-tips} by
% describing how tiles can be disassembled into shards,
% and how a sequence of tiles and shards is parseable
% into a well-formed term provided they ``fit'' together;
% in other words, \ty~ edit states and actions need not
% deal directly with the term syntax to enforce well-formedness.
% Having introduced the ingredients necessary for subjects,
% we will then describe frames in
% Section ?? and how they, too, can be disassembled and
% reassembled as needed.
% Finally, Section ?? will give the punchline.
% \note{revisit this paragraph and rework to more explicitly
% note subject/frame organization, also mention zipping}

% In particular, they have a \emph{bottom-up} zipper structure---consisting
% of a focused substructure $\subject$ and, separately, its surrounding
% ``inside-out'' context $\zframe$---

% \note{add some high level preview about edit states, zippers,
% justify this organization so that people are happy to wait;
% consider giving an example right away as I discuss bottom-up zippers}

% Parsing in \ty~ is stratified into two phases:
% assembling shards into tiles, and precedence-parsing
% tiles into terms.
% We will begin in Section \ref{sec:terms-tiles-tips}
% by showing that the precedence-parsing phase is
% guaranteed to succeed provided each element of the
% input sequence ``fits'' its neighbors.
% \note{sort of misleading, ignores children}
% As a result, we can structure \ty~ edit states using
% tiles rather than terms (Section \ref{sec:edit-states}),
% and the actions operating on edit states need only
% concern themselves with shard assembly and fixing holes
% to maintain sequential ``fit'' (Section \ref{sec:actions}).

% \note{something about how both of these parsing steps
% deviate from the usual approach of parsing, which takes
% a sequence of characters or tokens and produces a
% parse tree, whereas here we have more fine-grained
% delineation between different parseable/parsed structures,
% which is why we take care in this section to define
% derivations from one structure to another}

\subsection{Shards and pieces} \label{sec:shards-and-pieces}

The syntax for shards is given below.
\begin{figure}[h]
  \[\arraycolsep=4pt\begin{array}{rlrl}
    \text{pattern shard} & \shard^{\pat} & ::= &
      \shardlit{\texttt{(}} ~\vert~
      \shardlit{\texttt{)}} \\
    \text{expression shard} & \shard^{\expr} & ::= &
      \shardlit{\texttt{(}} ~\vert~
      \shardlit{\texttt{)}} ~\vert~
      \shardlit{\lambda} ~\vert~
      \shardlit{\texttt{.}} ~\vert~
      \shardlit{\texttt{let}} ~\vert~
      \shardlit{\texttt{=}} ~\vert~
      \shardlit{\texttt{in}}
  \end{array}\]
\end{figure}


\noindent \note{give an example of shard, note that
we don't model all shards that show up in overview and
that we'll discuss this when we talk about extensions}

Collectively we refer to shards and tiles as \emph{pieces}.
\begin{figure}[h]
  \[\arraycolsep=4pt\begin{array}{rlrl}

    \text{pieces} & \selection & ::= &
    \selem_1\dots\selem_n \\
    \text{piece} & \selem & ::= &
      \tile ~\vert~
      \shard \\

    \text{tile} & \tile & ::= &
       \tile^{\pat} ~\vert~
       \tile^{\expr} \\

    \text{shard} & \shard & ::= &
      \shard^{\pat} ~\vert~
      \shard^{\expr} \\
  \end{array}\]
\end{figure}


\subsection{Tips and terms} \label{sec:tips-and-terms}

\begin{figure}
  \judgbox{\flattenTerm{p}{\tiles^{\pat}}}{$p$ flattens to $\tiles^{\pat}$}
  \begin{mathpar}
    \inferrule[]{
    }{
      \flattenTerm{\term{\ophole}}{\optile{\ophole}}
    }\hspace{20pt}
    \inferrule[]{
    }{
      \flattenTerm{\term{\svar{x}}}{\optile{\svar{x}}}
    }\hspace{20pt}
    \inferrule[]{
      \flattenTerm{p}{\tiles^{\pat}}
    }{
      \flattenTerm{\term{\sparen{p}}}{\optile{\sparen{\tiles^{\pat}}}}
    }\\
    \inferrule[]{
      \flattenTerm{p_1}{\tiles^{\pat}_1} \\
      \flattenTerm{p_2}{\tiles^{\pat}_2}
    }{
      \flattenTerm{\term{\sbinhole{p_1}{p_2}}}{\tiles^{\pat}_1~\bintile{\binhole}~\tiles^{\pat}_2}
    }\hspace{20pt}
    \inferrule[]{
      \flattenTerm{p_1}{\tiles^{\pat}_1} \\
      \flattenTerm{p_2}{\tiles^{\pat}_2}
    }{
      \flattenTerm{\term{\sprod{p_1}{p_2}}}{\tiles^{\pat}_1~\bintile{\tprod}~\tiles^{\pat}_2}
    }
  \end{mathpar}

  \vspace{10pt}
  \judgbox{\flattenTerm{e}{\tiles^{\expr}}}{$e$ flattens to $\tiles^{\expr}$}
  \begin{mathpar}
    \inferrule[]{
    }{
      \flattenTerm{\term{\ophole}}{\optile{\ophole}}
    }\hspace{15pt}
    \inferrule[]{
    }{
      \flattenTerm{\term{\snumlit{n}}}{\optile{\snumlit{n}}}
    }\hspace{15pt}
    \inferrule[]{
    }{
      \flattenTerm{\term{\svar{x}}}{\optile{\svar{x}}}
    }\hspace{15pt}
    \inferrule[]{
      \flattenTerm{e}{\tiles^{\expr}}
    }{
      \flattenTerm{\term{\sparen{e}}}{\optile{\sparen{\tiles^{\expr}}}}
    }\\
    \inferrule[]{
      \flattenTerm{e_1}{\tiles^{\expr}_1} \\
      \flattenTerm{e_2}{\tiles^{\expr}_2}
    }{
      \flattenTerm{\term{\sbinhole{e_1}{e_2}}}{\tiles^{\expr}_1~\bintile{\binhole}~\tiles^{\expr}_2}
    }\hspace{20pt}
    \inferrule[]{
      \flattenTerm{e_1}{\tiles^{\expr}_1} \\
      \flattenTerm{e_2}{\tiles^{\expr}_2}
    }{
      \flattenTerm{\term{\sprod{e_1}{e_2}}}{\tiles^{\expr}_1~\bintile{\tprod}~\tiles^{\expr}_2}
    }\\
    \inferrule[]{
      \flattenTerm{p}{\tiles^{\pat}} \\
      \flattenTerm{e}{\tiles^{\expr}}
    }{
      \flattenTerm{\term{\slam{p}{e}}}{\pretile{\tlam{\tiles^{\pat}}}~\tiles^{\expr}}
    }\\
    \inferrule[]{
      \flattenTerm{p}{\tiles^{\pat}} \\
      \flattenTerm{e_1}{\tiles^{\expr}_1} \\
      \flattenTerm{e_2}{\tiles^{\expr}_2}
    }{
      \flattenTerm{\term{\slet{p}{e_1}{e_2}}}{\pretile{\tlet{\tiles^{\pat}}{\tiles^{\expr}_1}}~\tiles^{\expr}_2}
    }
  \end{mathpar}

  \caption{
    Term flattening
  }
  \label{fig:flatten-term}
  \end{figure}
\begin{figure}
  \vspace{-3px}
  \[
  \arraycolsep=3pt\begin{array}{rlrl}
      \mathsf{LeftTip} & \lTip & ::= & \ltipconv{s} ~\vert~ \ltipconc{s} \\
      \mathsf{RightTip} & \rTip & ::= & \rtipconv{s} ~\vert~ \rtipconc{s}
  \end{array}\]

  \judgbox{
    \fits{\rTip}{\lTip}
  }{
    Right tip $\rTip$ fits with left tip $\lTip$
  }
  \begin{mathpar}
    \inferrule[]{
    }{
      \fits{\rtipconc{s}}{\ltipconv{s}}
    } \\
    \inferrule[]{
    }{
      \fits{\rtipconv{s}}{\ltipconc{s}}
    }
  \end{mathpar}

  \caption{
    Syntax of tile and token tips
  }
  \label{fig:tip-syntax}
\end{figure}


\begin{definition}
  A selection $\selection$ is an \emph{opseq of sort $s$} if
  it is intact and $\lltip{s}~\rrtip{s}$-connected.
\end{definition}

\begin{theorem}
  A selection $\selection$ is an opseq of sort $s$
  if and only if there exists a term $x$ of sort $s$ such that
  $\flattenTerm{x}{\selection}$.
  \note{todo: sort-indexed term notation}
\end{theorem}

Finally we establish the connection between
\ty~ edit states and \emph{terms} of the
underlying language, as are typically defined
with a strictly tree-structured abstract syntax.
Figure \note{ref} shows the abstract syntax of
\ty~ language terms;
this syntax coincides with that in Figure \note{ref}
except for the inclusion of a new operator hole term
to account for operator holes in the tile syntax.

Figure \note{ref} defines the judgments
$\flattenTerm{p}{\tiles^{\pat}}$
and
$\flattenTerm{e}{\tiles^{\expr}}$,
which specify how to ``flatten'' a tree-structured
term into a corresponding sequence of tiles.
We consider a tile sequence $\tiles^s$ to be well-formed
with respect to the underlying language if there exists
a language term that flattens to $\tiles^s$.
\note{talk a bit about the parentheses rule, note that this is just for stating theorem}

Showing the existence of such a language term
is equivalent to showing the tiles in $\tiles^s$
form a valid operator sequence.
Each tile may be interpreted as a component
of an operator sequence, taking one of four
different shapes: operand, unary prefix,
unary suffix, and binary infix...



\subsection{Edit states} \label{sec:edit-states}
\begin{figure}
  \vspace{-3px}
  \[
  \arraycolsep=4pt\begin{array}{rlrl}
    \text{zipper} & \zip & ::= & \zipper{\subject}{\zframe} \\
    \text{subject} & \subject & ::= &
      \pointing{\selection}{\selection} ~\vert~
      \selecting{\selection}{\selection}{\selection} ~\vert~
      \restructuring{\selection}{\selection}{\selection} \\
    \text{frame} & \zframe & ::= &
      \tfrelem^{\pat} ~\vert~
      \tfrelem^{\expr} \\\\

    \text{sequence frame} & \tframe^s & ::= & \tframelit{\tiles^s\_\tiles^s}{\tfrelem^s} \\
    \text{pattern tile frame} & \tfrelem^{\pat} & ::= &
      \tframelit{\sparen{\_}}{\tframe^{\pat}} \\
    & & \vert &
      \tframelit{\slam{\_}{}}{\tframe^{\expr}} \\
    & & \vert &
      \tframelit{\slet{\_}{\tiles^{\expr}}{}}{\tframe^{\expr}} \\
    \text{expression tile frame} & \tfrelem^{\expr} & ::= &
      \tframelit{\sparen{\_}}{\tframe^{\expr}} \\
    & & \vert &
      \tframelit{\slet{\tiles^{\pat}}{\_}{}}{\tframe^{\expr}} \\\\
      % \scond{}{\tiles^{\expr}}{}

    \text{selection} & \selection & ::= &
    \selem_1\dots\selem_n \\
    \text{selected element} & \selem & ::= &
      \tile^{\pat} ~\vert~
      \tile^{\expr} ~\vert~
      \token^{\pat} ~\vert~
      \token^{\expr} \\\\

    \text{tile sequence} & \tiles^s & ::= & \tile^s_1\dots\tile^s_n \\
    \text{pattern tile} & \tile^{\pat} & ::= &
      \ophole ~\vert~
      \svar{x} ~\vert~
      % \sann{}{\tiles^{\typ}} ~\vert~
      \binhole ~\vert~
      \sprod{}{} ~\vert~
      \sparen{\tiles^{\pat}}\\
    \text{expression tile} & \tile^{\expr} & ::= &
      \ophole ~\vert~
      % \sboollit{b} ~\vert~
      \snumlit{n} ~\vert~
      \svar{x} ~\vert~
      % \sap{}{} ~\vert~
      \binhole ~\vert~
      \sprod{}{} ~\vert~
      \splus{}{} ~\vert~
      \smult{}{} ~\vert~
      % \sequals{}{} ~\vert~
      \sparen{\tiles^{\expr}} \\
    & & \vert &
      \slam{\tiles^{\pat}}{} ~\vert~
      \slet{\tiles^{\pat}}{\tiles^{\expr}}{}\\\\ % ~\vert~
      % \scond{}{\tiles^{\expr}}{}

      \text{pattern token} & \token^{\pat} & ::= &
        \tokenLit{\texttt{(}} ~\vert~
        \tokenLit{\texttt{)}} \\
      \text{expression token} & \token^{\expr} & ::= &
        \tokenLit{\texttt{(}} ~\vert~
        \tokenLit{\texttt{)}} ~\vert~
        \tokenLit{\lambda} ~\vert~
        \tokenLit{\texttt{.}} ~\vert~
        \tokenLit{\texttt{let}} ~\vert~
        \tokenLit{\texttt{=}} ~\vert~
        \tokenLit{\texttt{in}}
  \end{array}\]
  \caption{
    Syntax of edit states \note{consider moving tiles and tokens to overview section}
  }
  \label{fig:edit-state-syntax}
\end{figure}

\note{set up running example right away}

We start with an overview of the syntax of \ty~ edit states,
presented altogether in Figure \ref{fig:edit-state-syntax}.
Edit states in \ty~ are sort-indexed structures
$\zipper{\subject}{\tfrelem^s}$ that follow a variant
of the zipper pattern first described by Huet.
In particular, they have a \emph{bottom-up}
zipper structure \note{cite}---consisting of a focused substructure
$\subject$ and, separately,
its surrounding ``inside-out'' context $\tfrelem^s$---unlike
the top-down structure adopted in other work \note{cite}.
In this work, we call the focused substructure the \emph{subject} of
the zipper and its surrounding context the \emph{frame}.

\subsubsection{Tiles, tokens, \& selections}
The subject $\sigma$ of a zipper $\zipper{\subject}{\tfrelem^s}$
may be in one of three modes.
\note{name them here}
We defer higher-level discussion regarding their roles in
editing to Section \note{ref}, where we define \ty's editing
operations, and focus now on the structures that constitute them.

% The first mode, called \emph{pointing mode}, consists of
% a pair of tiles of the sort expected by the frame.
% We described the syntax of tiles
% in Section \ref{sec:overview};
% the same syntax that was presented in Figure \note{ref}
% is presented again in Figure \note{ref} in context.

% The second and third modes, called \emph{selecting mode}
% and \emph{restructuring mode}, each consist of a trio of
% \emph{selections}.

\note{add back some discussion of tiles/tokens here}

All three modes consist of two or more \emph{selections}.
A selection is a linear sequence of heterogeneously
sorted \emph{tiles} and \emph{tokens}.
We described the syntax of tiles $\tile^s$
in Section \ref{sec:overview};
the same syntax that was presented in Figure \ref{fig:tile-syntax}
is presented again in Figure \ref{fig:edit-state-syntax} in context.
Tokens $\shard^s$ are the lexical components of tiles, forming either
the substance of childless tiles or the delimiters of parent
tiles' children.
Tokens are generated from tiles via the \emph{tile disassembly} function
$\disassembleTile{\cdot}$, defined in Figure \ref{fig:disassemble-tile},
which takes a tile and produces
a selection consisting of the tile's tokens
and the tile's children tiles.
\note{need more examples}

\note{find different phrasing here that doesn't invite interpretation as technical terms}
Selections have a permissive syntax, with multiple
ways of representing the same essential content.
For example, both a singleton tile and the tile's disassembly
form valid selections.
\note{make this a real example}
Such a permissive syntax is useful for representing
arbitrary divisions of a program's
lexical components, but we would like to ensure
that the divided selections remain ``maximally assembled''
in order to inform the user of the top-level structures
they are manipulating.
\note{give explicit examples of divisions showing maximally
and non-maximally assembled selections}

This calls for defining an appropriate
parser on selections to maintain
maximal structure as they grow and shrink during editing.
\note{make this more precise "when the selection changes",
perhaps add another example here of a selection growing}
Such a parser cannot be specified in the standard way
with a context-free grammar because such parsers are
all-or-nothing, whereas in our case our parser should
produce structure opportunistically while not requiring any
specific top-level form.
\note{maybe don't talk about this here, if anything talk about
it after I define what we did}
Much as a CFG-based parser identifies a derivation from
the CFG's start symbol to the input string,
our selection parser identifies an analogous ``disassembly'' from
a maximally structured selection to the input selection.
\note{more example}
Figure \ref{fig:disassemble-selection} defines the step-disassembly relation $\selection_1\searrow\selection_2$,
which chooses a tile in $\selection_1$ and replaces it with the
result of its disassembly.
We can show that the reflexive transitive
closure $\searrow^*$ of $\searrow$, called \emph{selection disassembly},
is a partial order in
which every selection has a unique maximal element that
disassembles to it.
\begin{lemma}
  $\searrow^*$ is a partial order.
\end{lemma}
\begin{lemma}\label{lemma:unique-parsed-selection}
  For every selection $\selection$, there is a unique
  selection $\selection'$ such that:
  \begin{itemize}
  \item $\selection'\searrow^*\selection$, and
  \item $\selection'$ is maximal: if $\selection''\searrow^*\selection'$ then $\selection'' = \selection'$.
  \end{itemize}
\end{lemma}
Lemma \ref{lemma:unique-parsed-selection} makes it
possible to specify a well-defined \emph{selection assembly}
function:
\begin{definition}
  Given a selection $\selection$, let $\parseSelection{\selection}$ be
  the unique maximal selection that disassembles down
  to $\selection$.
\end{definition}
We defer presenting a constructive implementation of this function
to the imaginary appendix.

\begin{figure}
  \newcommand{\dd}{\searrow}
  \judgbox{\disassemblesDown{\selem}{\selection}}{$\selem$ disassembles down to $\selection$}
  \[
    \arraycolsep=3pt
    \begin{array}{rcl}
      \sparen{\tiles^s} & \dd & \shardlit{\texttt{(}}~\tiles^s~\shardlit{\texttt{)}} \\
      \slam{\tiles^{\pat}} & \dd & \shardlit{\lambda}~\tiles^{\pat}~\shardlit{\texttt{.}} \\
      \slet{\tiles^{\pat}}{\tiles^{\expr}}{} & \dd & \shardlit{\texttt{let}}~\tiles^{\pat}~\shardlit{\texttt{=}}~\tiles^{\expr}~\shardlit{\texttt{in}}
  \end{array}\]
  \caption{
    Disassembling a selected element $\selem$.
    If $\selem$ has a syntactic form not listed, then we assume that $\disassemblesDown{\selem}{\cdot}$.
  }
  \label{fig:disassemble-tile}
\end{figure}
\begin{figure}
\judgbox{\stepDisassembleSelection{\selection_1}{\selection_2}}{$\selection_1$ disassembles to $\selection_2$}
\begin{mathpar}
  \inferrule[]{
    \disassembleTile{\tau^s} = \Psi'
  }{
    \stepDisassembleSelection{\tau^s\Psi}{\Psi'\Psi}
  }\hspace{40pt}
  \inferrule[]{
    \stepDisassembleSelection{\Psi}{\Psi'}
  }{
    \stepDisassembleSelection{\psi\Psi}{\psi\Psi'}
  }
\end{mathpar}
\caption{
  Disassembling a selection
}
\label{fig:disassemble-selection}
\end{figure}

\subsubsection{Frames \& zippers}
The sort-indexed frame $\tfrelem^s$ of a zipper
$\zipper{\subject}{\tfrelem^s}$ models the rest of
the program surrounding the subject $\subject$,
the sort index $s$ specifying what sort is expected
of the subject.
Frames are nested in bottom-up fashion, starting
with the nearest containing structure and concluding
with the program root.
For example, if the edit subject is located at $\Box$
in the program \texttt{let x = ( $\Box$ ) + 1 in x + 2},
then the frame is represeted in our syntax as
\[
  \tframelit{
    \sparen{\_}
  }{
    \tframelit{
      \_\splus{}{\texttt{1}}
    }{
      \tframelit{
        \slet{\texttt{x}}{\_}{}
      }{
        \tframelit{
          \_\ \splus{\texttt{x}}{\texttt{2}}
        }{
          \note{root}
        }
      }
    }
  }
\]
Notice that frames alternate between \emph{tile
frames} $\tfrelem^s$ that form bidelimited containers and
\emph{sequence frames} $\tframe^s$ consisting of the tiles
surrounding each tile frame.

% Having defined subjects and frames, we may now
% define the syntax of edit states,
% shown in Figure \note{ref}.
% An edit state consists of a sort-indexed \emph{zipper},
% each consisting of a subject and a frame of the
% specified sort.

% \begin{figure}
  \vspace{-3px}
  \[
  \arraycolsep=3pt\begin{array}{rlrl}
      \mathsf{TilesFrame} & \tframe^s & ::= & \tframelit{\tiles^s\_\tiles^s}{\tfrelem^s} \\
      % \mathsf{Tile}^{\typ} & \tile^{\typ} & ::= &
      %     % \tnum ~\vert~
      %     \shole ~\vert~
      %     \sbool ~\vert~
      %     \snum ~\vert~
      %     \sarr{}{} ~\vert~
      %     \sprod{}{} ~\vert~
      %     \sparen{\tiles^{\typ}}\\
      \mathsf{PatTileFrame} & \tfrelem^{\pat} & ::= &
        \tframelit{\sparen{\_}}{\tframe^{\pat}} \\
      & & \vert &
        \tframelit{\slam{\_}{}}{\tframe^{\expr}} \\
      & & \vert &
        \tframelit{\slet{\_}{\tiles^{\expr}}{}}{\tframe^{\expr}} \\
      \mathsf{ExpTileFrame} & \tfrelem^{\expr} & ::= &
        \tframelit{\sparen{\_}}{\tframe^{\expr}} \\
      & & \vert &
        \tframelit{\slet{\tiles^{\pat}}{\_}{}}{\tframe^{\expr}}
        % \scond{}{\tiles^{\expr}}{}
  \end{array}\]
  \caption{
    Syntax of pattern and expression frames.
  }
  \label{fig:tile-syntax}
\end{figure}

% \begin{figure}
  \[\arraycolsep=3pt\begin{array}{rlrl}
      \text{zipper} & \zip & ::= & \zipper{\subject}{\zframe} \\
      \text{subject} & \subject & ::= &
        \pointing{\selection}{\selection} ~\vert~
        \selecting{\selection}{\selection}{\selection} ~\vert~
        \restructuring{\selection}{\selection}{\selection} \\
      \text{frame} & \zframe & ::= &
        \tfrelem^{\pat} ~\vert~ \tfrelem^{\expr}
 \end{array}\]
  \caption{
    Zipper syntax
  }
  \label{fig:zipper-syntax}
\end{figure}

\begin{figure}
  \vspace{-3px}
  \[
  \setlength{\fboxsep}{1pt}
  \arraycolsep=3pt\def\arraystretch{1.4}\begin{array}{rcl}
      \disassembleTileFrame{
        \tframelit{
          \sparen{\_}
        }{
          \tframelit{\tiles^s_1\_\tiles^s_2}{\tfrelem^{\pat}}
        }
      } & = &
        \zipper{
          \pointing{\tiles^s_1\tokenLit{\texttt{(}}~}{~\tokenLit{\texttt{)}}\tiles^s_2}
        }{
          \tfrelem^{\pat}
        } \\

      \disassembleTileFrame{
        \tframelit{
          \slam{\_}{}
        }{
          \tframelit{\tiles^{\expr}_1\_\tiles^{\expr}_2}{\tfrelem^{\expr}}
        }
      } & = &
        \zipper{
          \pointing{\tiles^{\expr}_1\tokenLit{\lambda}~}{~\tokenLit{\texttt{.}}\tiles^{\expr}_2}
        }{
          \tfrelem^{\expr}
        } \\

      \disassembleTileFrame{
        \tframelit{
          \slet{\_}{\tiles^{\expr}_0}{}
        }{
          \tframelit{\tiles^{\expr}_1\_\tiles^{\expr}_2}{\tfrelem^{\expr}}
        }
      } & = & \\
        \zipper{
          \pointing{
            \tiles^{\expr}_1\tokenLit{\texttt{let}}~
          }{
            ~\tokenLit{\texttt{=}}~\tiles^{\expr}_0\tokenLit{\texttt{in}}~\tiles^{\expr}_2
          }
        }{
          \tfrelem^{\expr}
        } \\

      \disassembleTileFrame{
        \tframelit{
          \slet{\tiles^{\pat}}{\_}{}
        }{
          \tframelit{\tiles^{\expr}_1\_\tiles^{\expr}_2}{\tfrelem^{\expr}}
        }
      } & = & \\
        \zipper{
          \pointing{
            \tiles^{\expr}_1\tokenLit{\texttt{let}}~\tiles^{\pat}~\tokenLit{\texttt{=}}~
          }{
            ~\tokenLit{\texttt{in}}~\tiles^{\expr}_2
          }
        }{
          \tfrelem^{\expr}
        }
  \end{array}\]
  \caption{
    Disassembling a tile frame.
  }
  \label{fig:disassemble-tile}
\end{figure}
\begin{figure}
  \judgbox{\disassembleZipper{\editState_1}{\editState_2}}{$\editState_1$ step-disassembles up to $\editState_2$}
  \begin{mathpar}
    \inferrule[]{
      \disassemblesUp{\zframe}{\zipper{\pointing{\selection_3}{\selection_4}}{\zframe'}}
    }{
      \disassembleZipper{
        \zipper{\pointing{\selection_1}{\selection_2}}{\zframe}
      }{
        \zipper{\pointing{\selection_3\selection_1}{\selection_2\selection_4}}{\zframe'}
      }
    }
  \end{mathpar}
  \caption{
    Step-disassembly of a zipper
  }
  \label{fig:disassemble-zipper}
  \end{figure}

Like tiles, frames may be decomposed into their
lexically constituent tiles and tokens via the
\emph{frame disassembly} function $\disassembleTileFrame{\cdot}$,
defined in Figure \note{ref}.
Similar to how we lift tile disassembly to selection
disassembly, we can lift frame disassembly to
\emph{edit state step-disassembly}, defined in Figure \note{ref}.
We can show that the reflexive transitive closure
$\nearrow^*$ of $\nearrow$, which we call \emph{edit state disassembly},
is a partial order in
which every edit state with a subject in pointing mode
has a unique minimal element that disassembles up to it:
\begin{lemma}
  $\nearrow^*$ is a partial order.
\end{lemma}
\begin{lemma}\label{lemma:unique-parsed-editstate}
  For every edit state $\editState = \zipper{\pointing{\selection_1}{\selection_2}}{\tfrelem^s}$,
  there exists a unique $\editState' = \zipper{\pointing{\selection'_1}{\selection'_2}}{\tfrelem^{s'}}$ such that
  \begin{itemize}
  \item $\editState' \nearrow^* \editState$, and
  \item if $\editState'' \nearrow^* \editState'$ then $\editState'' = \editState'$.
  \end{itemize}
\end{lemma}
Lemma \ref{lemma:unique-parsed-editstate} makes it
possible to specify a well-defined parsing function
on edit states in pointing mode:
\begin{definition}
  Given an edit state $\editState = \zipper{\pointing{\selection_1}{\selection_2}}{\tfrelem^s}$,
  let $\parseZipper{\editState}$ be the unique minimal edit state
  that disassembles up to $\editState$.
\end{definition}

% \begin{definition}
%   $\zeta^s_1\nearrow\zeta^s_2$ if $\disassembleTileFrame{\zeta^s_1} = \zeta^s_2$.
%   \note{defined this for symmetry with diassembling of tiles}
% \end{definition}

% \begin{definition}
%   $\nearrow^*$ is the reflexive transitive closure of $\nearrow$.
% \end{definition}

\subsection{Actions} \label{sec:actions}

\note{add invariants for all modes, or consider
modifying restructuring movement to take into account
sort (but emphasize what is obligatory as you scale)}

We now present \ty's central action judgment
$\performAction{\editState_1}{\action}{\editState_2}$,
which sends edit state $\editState_1$ to $\editState_2$
via action $\action$.
The action judgment is governed by a syntactic sensibility
theorem (Theorem \note{ref}) that ensures that
the subject of every action-reachable edit state can
be parsed into language term.

% \begin{figure}
  \newcommand{\spacing}{\ \ \ \ \ }
  \[
  \setlength{\fboxsep}{1pt}
  \arraycolsep=3pt\def\arraystretch{1.25}\begin{array}{lcll}
    % \fixHolesFn{\selection_1}{\selection_2} & = &
    %   \begin{cases}
    %     (~\cdot~, \ophole\selection_2) & \text{if } \selection_1 =~\cdot~ \text{ and } \leftTip{\selection_2} =\ \rtip
    %   \end{cases} \\
      \fixHolesFn{~\cdot~}{\selem\selection} & = &
        (~\cdot~, \ophole\selem\selection) & \text{\spacing if } \leftTip{\selem} =\ \rtip \\
     \fixHolesFn{\selection\selem}{~\cdot~} & = &
       (\selection\selem\ophole, ~\cdot~) & \text{\spacing if } \rightTip{\selem} = \ltip \\
    \fixHolesFn{\selection\ophole}{\selem\selection'} & = &
      (\selection, \selem\selection') & \text{\spacing if } \leftTip{\selem} = \ltip \\
   \fixHolesFn{\selection\selem}{\ophole\selection'} & = &
     (\selection\selem, \selection') & \text{\spacing if } \rightTip{\selem} =\ \rtip \\
  \fixHolesFn{\selection\binhole}{\selem\selection'} & = &
      (\selection, \selem\selection') & \text{\spacing if } \leftTip{\selem} =\ \rtip \\
  \fixHolesFn{\selection\selem}{\binhole\selection'} & = &
    (\selection\selem, \selection') & \text{\spacing if } \rightTip{\selem} = \ltip \\
  \fixHolesFn{\selection\selem}{\selem'\selection'} & = &
      (\selection\selem, \ophole\selem'\selection') & \text{\spacing if } \rightTip{\selem} = \ltip \text{ and }\leftTip{\selem'} =\ \rtip \\
  \fixHolesFn{\selection\selem}{\selem'\selection'} & = &
    (\selection\selem, \binhole\selem'\selection') & \text{\spacing if } \rightTip{\selem} =\ \rtip \text{ and }\leftTip{\selem'} = \ltip \\
  \fixHolesFn{\selection}{\selection'} & = & (\selection, \selection') & \text{\spacing otherwise}
\end{array}\]
  \vspace{-2px}
  \CaptionLabel{Hole fixing}{fig:hole-fixing}
  \vspace{-2px}
  \end{figure}
\begin{figure}
  \judgbox{
    \fixHolesSelection{\rTip_1}{\selection_1}{\selection_2}{\rTip_2}
  }{
    $\selection_1$ is hole fixed under tip constraint $\rTip_1$ \\
    to produce $\selection_2$ and new tip constraint $\rTip_2$
  }
  \begin{mathpar}
    \inferrule[]{
    }{
      \fixHolesSelection{r}{\cdot}{\cdot}{r}
    } \\
    \inferrule[]{
      \text{\note{$\psi$ is a hole}} \\
      \fixHolesSelection{r}{\selection}{\selection'}{r'}
    }{
      \fixHolesSelection{r}{\selem\selection}{\selection'}{r'}
    } \\
    \inferrule[]{
      \text{\note{$\psi$ not a hole}} \\
      \fits{r}{\leftTip{\selem}} \\
      \fixHolesSelection{\rightTip{\selem}}{\selection}{\selection'}{r'}
    }{
      \fixHolesSelection{r}{\selem\selection}{\selection'}{r'}
    } \\
    \inferrule[]{
      \text{\note{$\psi$ not a hole}} \\
      \leftTip{\selem} = \ltipconc{s} \\
      \fixHolesSelection{\rightTip{\selem}}{\selection}{\selection'}{r'}
    }{
      \fixHolesSelection{\rtipconc{s}}{\selem\selection}{\ophole^s\selem\selection'}{r'}
    } \\
    \inferrule[]{
      \text{\note{$\psi$ not a hole}} \\
      \leftTip{\selem} = \ltipconv{s} \\
      \fixHolesSelection{\rightTip{\selem}}{\selection}{\selection'}{r'}
    }{
      \fixHolesSelection{\rtipconv{s}}{\selem\selection}{\binhole^s\selem\selection'}{r'}
    } \\
  \end{mathpar}

  \judgbox{
    \fixHolesSelections{\rTip}{\selection_1}{\selection_2}{\lTip}{\selection_3}{\selection_4}
  }{$\selection_1$ and $\selection_2$ are hole fixed under \\
    tip constraints $\rTip$ and $\lTip$ \\
    to produce $\selection_3$ and $\selection_4$
  }
  \vspace{10pt}
  \begin{mathpar}
    \inferrule[]{
      \fixHolesSelection{r}{\selection_1}{\selection'_1}{r'} \\
      \fixHolesSelection{r'}{\selection_2}{\selection'_2}{r''} \\
      \fits{r''}{l}
    }{
      \fixHolesSelections{r}{\selection_1}{\selection_2}{l}{\selection'_1}{\selection'_2}
    } \\
    \inferrule[]{
      \fixHolesSelection{r}{\selection_1}{\selection'_1}{r'} \\
      \fixHolesSelection{r'}{\selection_2}{\selection'_2}{\rtipconc{s}}
    }{
      \fixHolesSelections{r}{\selection_1}{\selection_2}{\ltipconc{s}}{\selection'_1}{\selection'_2\ophole^s}
    }
  \end{mathpar}
  \caption{
    Hole fixing
  }
  \label{fig:fixholes-2}
  \end{figure}

\begin{figure}
  \newcommand{\spacing}{\ \ \ \ \ }
  \[
  \setlength{\fboxsep}{1pt}
  \arraycolsep=3pt\def\arraystretch{1.25}\begin{array}{lcl}
    % \fixHolesFn{\selection_1}{\selection_2} & = &
    %   \begin{cases}
    %     (~\cdot~, \ophole\selection_2) & \text{if } \selection_1 =~\cdot~ \text{ and } \leftTip{\selection_2} =\ \rtip
    %   \end{cases} \\
      \filterTiles{s}{\cdot} & = & \cdot \\
      \filterTiles{s}{\token^{s'}\selection} & = & \filterTiles(\selection) \\
      \filterTiles{s}{\tile^{s'}\selection} & = &
        \begin{cases}
          \tile^{s'}\filterTiles{s}{\selection} & \text{ if } s = s' \\
          \filterTiles{s}{\selection} & \text{ else }
        \end{cases}
\end{array}\]
  \vspace{-2px}
  \CaptionLabel{Filtering tiles}{fig:filter-tiles}
  \vspace{-2px}
  \end{figure}
\begin{figure}
  \vspace{-3px}
  \judgbox{
    \wholeSelection{s}{\selection}
  }{
    $\selection$ consists of tiles of sort $s$
  }
  \begin{mathpar}
    \inferrule[]{
    }{
      \wholeSelection{s}{\cdot}
    } \ \ \ \ \ \ \ \ \ \
    \inferrule[]{
      \wholeSelection{s}{\selection}
    }{
      \wholeSelection{s}{\tile^s\selection}
    }
  \end{mathpar}
  \caption{
    Whole selections
  }
  \label{fig:whole-selection}
\end{figure}

\begin{figure}
  \vspace{-3px}
  \[
  \arraycolsep=3pt\begin{array}{rlrl}
      \mathsf{Action} & \action & ::= &
        \actionlit{mark} ~\vert~
        \actionlit{move}~\direction ~\vert~
        \actionlit{delete} ~\vert~
        \actionlit{construct}~\tile^s \\
      \mathsf{Direction} & \direction & ::= &
        \texttt{left} ~\vert~
        \texttt{right}
  \end{array}\]
  \caption{Syntax of actions}
  \label{fig:action-syntax}
\end{figure}


\begin{figure}
  \judgbox{\pmove{\editState_1}{\direction}{\editState_2}}{$\editState_1$ transitions via movement $\direction$ to $\editState_2$\\\ \ in pointing mode}
  \begin{mathpar}
  \inferrule[PMoveRightAtomic]{
    \disassemblesDown{\selem}{\cdot}
  }{
    \pmove{
      \zipper{\pointing{\selection_1}{\selem\selection_2}}{\zframe}
    }{
      \texttt{right}
    }{
      \parseZipper{\zipper{\pointing{\parseSelection{\selection_1\selem}}{\selection_2}}{\zframe}}
    }
  }

  \inferrule[PMoveRightDisassembles]{
    \disassemblesDown{\selem}{\selection_3} \\
    \selection_3\neq\cdot \\
    \pmove{
      \zipper{\pointing{\selection_1}{\selection_3\selection_2}}{\zframe}
    }{
      \texttt{right}
    }{
      \zip
    }
  }{
    \pmove{
      \zipper{\pointing{\selection_1}{\selem\selection_2}}{\zframe}
    }{
      \texttt{right}
    }{
      \zip
    }
  }

  \inferrule[PMoveRightFrame]{
    \disassemblesUp{
      \zframe
    }{
      \zipper{\pointing{\selection_2}{\selection_3}}{\zframe'}
    } \\
    \pmove{
      \zipper{\pointing{\selection_2\selection_1}{\selection_3}}{\zframe'}
    }{
      \texttt{right}
    }{
      \zip
    }
  }{
    \pmove{
      \zipper{\pointing{\selection_1}{\cdot}}{\zframe}
    }{
      \texttt{right}
    }{
      \zip
    }
  }
\end{mathpar}

\vspace{-2px}
\CaptionLabel{Movement in pointing mode}{fig:move-pointing}
\vspace{-2px}
\end{figure}

\begin{figure}
  \judgbox{\smove{\editState_1}{\direction}{\editState_2}}{$\editState_1$ moves to $\editState_2$ in selecting mode}
  \begin{mathpar}

\inferrule[SMoveLeftAtomic]{
  \nsegmentdd{\selem} \\
  \parseSelection{\selem\selection_2} = \selection'_2
}{
  \smove{
    \zipper{\selecting{\selection_1\selem}{\selection_2}{\selection_3}}{\zframe}
  }{
    \texttt{left}
  }{
    \zipper{\selecting{\selection_1}{\selection'_2}{\selection_3}}{\zframe}
  }
}

\inferrule[SMoveLeftDisassembles]{
  \pieceDisassembles{\selem}{\selection_4} \\
  \smove{
    \zipper{\selecting{\selection_1\selection_4}{\selection_2}{\selection_3}}{\zframe}
  }{
    \texttt{left}
  }{
    \zip
  }
}{
  \smove{
    \zipper{\selecting{\selection_1\selem}{\selection_2}{\selection_3}}{\zframe}
  }{
    \texttt{left}
  }{
    \zip
  }
}

\inferrule[SMoveLeftFrame]{
  \framedu{
    \zframe
  }{
    \zipper{\pointing{\selection_3}{\selection_4}}{\zframe'}
  } \\
  \smove{
    \zipper{\selecting{\selection_3}{\selection_1}{\selection_2\selection_4}}{\zframe'}
  }{
    \texttt{left}
  }{
    \zip
  }
}{
  \smove{
    \zipper{\selecting{\cdot}{\selection_1}{\selection_2}}{\zframe}
  }{
    \texttt{left}
  }{
    \zip
  }
} \\

\inferrule[SMoveRightAtomic]{
  \nsegmentdd{\selem} \\
  \parseZipper{\zipper{\pointing{\parseSelection{\selection_1\selem}}{\selection_3}}{\zframe}} =
    \zipper{\pointing{\selection'_1}{\selection'_3}}{\zframe'}
}{
  \smove{
    \zipper{\selecting{\selection_1}{\selem\selection_2}{\selection_3}}{\zframe}
  }{
    \texttt{right}
  }{
    \zipper{\selecting{\selection'_1}{\selection_2}{\selection'_3}}{\zframe'}
  }
}

\inferrule[SMoveRightDisassembles]{
  \pieceDisassembles{\selem}{\selection_4} \\
  \smove{
    \zipper{\selecting{\selection_1}{\selection_4\selection_2}{\selection_3}}{\zframe}
  }{
    \texttt{right}
  }{
    \zip
  }
}{
  \smove{
    \zipper{\selecting{\selection_1}{\selem\selection_2}{\selection_3}}{\zframe}
  }{
    \texttt{right}
  }{
    \zip
  }
}

\end{mathpar}

\vspace{-2px}
\CaptionLabel{Movement in selecting mode}{fig:move-selecting}
\vspace{-2px}
\end{figure}


\begin{figure}
  \judgbox{\rmove{\editState_1}{\direction}{\editState_2}}{$\editState_1$ moves to $\editState_2$ in restructuring mode}
  \begin{mathpar}

\inferrule[RMoveRightCracked]{
  \text{$\selection_2$ is cracked}
}{
  \rmove{
    \zipper{\restructuring{\selection_1}{\selection_2}{\tile^{s}\selection_3}}{\zframe}
  }{
    \texttt{right}
  }{
    \zipper{\restructuring{\selection_1\tile^{s}}{\selection_2}{\selection_3}}{\zframe}
  }
}

\inferrule[RMoveRightIntact]{
  \text{$\selection_2$ is intact} \\
  \pmove{
    \zipper{\pointing{\selection_1}{\selection_3}}{\zframe}
  }{
    \texttt{right}
  }{
    \zipper{\pointing{\selection'_1}{\selection'_3}}{\zframe'}
  }
}{
  \rmove{
    \zipper{\restructuring{\selection_1}{\selection_2}{\selection_3}}{\zframe}
  }{
    \texttt{right}
  }{
    \zipper{\restructuring{\selection'_1}{\selection_2}{\selection'_3}}{\zframe'}
  }
} \\

\end{mathpar}

\vspace{-2px}
\CaptionLabel{Movement in restructuring mode}{fig:move-restructuring}
\vspace{-2px}
\end{figure}


\begin{figure*}
  \judgbox{\performAction{\editState_1}{\action}{\editState_2}}{$\editState_1$ transitions via $\alpha$ to $\editState_2$}
  \begin{mathpar}
  \inferrule[MoveRightPast]{
    \disassemblesDown{\selem}{\cdot}
  }{
    \performAction{
      \zipper{\pointing{\selection_1}{\selem\selection_2}}{\tfrelem^s}
    }{
      \actionlit{move}\ \texttt{right}
    }{
      \zipper{\pointing{\selection_1\selem}{\selection_2}}{\tfrelem^s}
    }
  } \\
  \inferrule[MoveRightEnter]{
    \disassemblesDown{\selem}{\selem'\selection_3}
  }{
    \performAction{
      \zipper{\pointing{\selection_1}{\selem\selection_2}}{\tfrelem^s}
    }{
      \actionlit{move}\ \texttt{right}
    }{
      \parseZipper{\zipper{\pointing{\selection_1\selem'}{\selection_3\selection_2}}{\tfrelem^s}}
    }
  } \ \ \ \ \ \
  \inferrule[MoveRightExitTile]{
    \disassemblesUp{
      \tfrelem^s
    }{
      \zipper{\pointing{\selection_2}{\selem\selection_3}}{\tfrelem^{s'}}
    }
  }{
    \performAction{
      \zipper{\pointing{\selection_1}{\cdot}}{\tfrelem^s}
    }{
      \actionlit{move}\ \texttt{right}
    }{
      \parseZipper{
        \zipper{\pointing{\selection_2\selection_1\selem}{\selection_3}}{\tfrelem^{s'}}
      }
    }
  } \\
  \inferrule[ConstructTile]{
    \disassemblesDown{\tile^s}{\token^s\selection_3} \\
    \fixHolesSelections{\rtipconc{s}}{\selection_1\token^s}{\selection_3\selection_2}{\ltipconc{s}}{\selection_4}{\selection_5}
  }{
    \performAction{
      \zipper{\pointing{\selection_1}{\selection_2}}{\tfrelem^s}
    }{
      \actionlit{construct}\ \tile^s
    }{
      \parseZipper{
        \zipper{\pointing{\selection_4}{\selection_5}}{\tfrelem^{s}}
      }
    }
  } \ \ \ \ \ \ \ \
  \inferrule[MarkPointing]{
  }{
    \performAction{
      \zipper{\pointing{\selection_1}{\selection_2}}{\tfrelem^s}
    }{
      \actionlit{mark}
    }{
      \zipper{\selecting{\selection_1}{\strut~\cdot~}{\selection_2}}{\tfrelem^s}
    }
  } \\
  \inferrule[SelectLeftAtom]{
    \disassemblesDown{\selem}{\cdot} \\
    \parseSelection{\selem\selection_2} = \selection'_2
  }{
    \performAction{
      \zipper{\selecting{\selection_1\selem}{\selection_2}{\selection_3}}{\tfrelem^s}
    }{
      \actionlit{move}\ \texttt{left}
    }{
      \zipper{\selecting{\selection_1}{\selection'_2}{\selection_3}}{\tfrelem^s}
    }
  } \ \ \
  \inferrule[SelectLeftDisassembles]{
    \disassemblesDown{\selem}{\selection_4\selem'} \\
    \parseSelection{\selem'\selection_2} = \selection'_2
  }{
    \performAction{
      \zipper{\selecting{\selection_1\selem}{\selection_2}{\selection_3}}{\tfrelem^s}
    }{
      \actionlit{move}\ \texttt{left}
    }{
      \zipper{\selecting{\selection_1\selection_4}{\selection'_2}{\selection_3}}{\tfrelem^s}
    }
  } \ \ \
  \inferrule[SelectLeftExit]{
    \disassemblesUp{
      \tfrelem^s
    }{
      \zipper{\pointing{\selection_3\selem}{\selection_4}}{\tfrelem^{s'}}
    }
  }{
    \performAction{
      \zipper{\selecting{\cdot}{\selection_1}{\selection_2}}{\tfrelem^s}
    }{
      \actionlit{move}\ \texttt{left}
    }{
      \zipper{\selecting{\selection_3}{\selection_1\selem}{\selection_2\selection_4}}{\tfrelem^{s'}}
    }
  } \\
  \inferrule[SelectRightAtom]{
    \disassemblesDown{\selem}{\cdot} \\
    \parseZipper{\zipper{\pointing{\selection_1\selem}{\selection_3}}{\tfrelem^s}} =
      \zipper{\pointing{\selection'_1}{\selection'_3}}{\tfrelem^{s'}}
  }{
    \performAction{
      \zipper{\selecting{\selection_1}{\selem\selection_2}{\selection_3}}{\tfrelem^s}
    }{
      \actionlit{move}\ \texttt{right}
    }{
      \zipper{\selecting{\selection'_1}{\selection_2}{\selection'_3}}{\tfrelem^{s'}}
    }
  } \ \ \ \ \ \ \ \
  \inferrule[SelectRightDisassembles]{
    \disassemblesDown{\selem}{\selem'\selection_4}
  }{
    \performAction{
      \zipper{\selecting{\selection_1}{\selem\selection_2}{\selection_3}}{\tfrelem^s}
    }{
      \actionlit{move}\ \texttt{right}
    }{
      \zipper{\selecting{\selection_1\selem'}{\selection_4\selection_2}{\selection_3}}{\tfrelem^s}
    }
  } \\
  \inferrule[MarkSelecting]{
    \matchesSort{\leftTip{\selection_2}}{\rightTip{\selection_2}} \\
    \fixHolesSelections{\rtipconc{s}}{\selection_1}{\selection_3}{\ltipconc{s}}{\selection'_1}{\selection'_3}
  }{
    \performAction{
      \zipper{\selecting{\selection_1}{\selection_2}{\selection_3}}{\tfrelem^s}
    }{
      \actionlit{mark}
    }{
      \zipper{\restructuring{\selection'_1}{\selection_2}{\selection'_3}}{\tfrelem^s}
    }
  } \\
  \inferrule[MoveRightRestructuringNotWhole]{
    \text{\note{not\ }}\wholeSelection{s'}{\selection_2}
  }{
    \performAction{
      \zipper{\restructuring{\selection_1}{\selection_2}{\tile^{s'}\selection_3}}{\tfrelem^s}
    }{
      \actionlit{move}\ \texttt{right}
    }{
      \zipper{\restructuring{\selection_1\tile^{s'}}{\selection_2}{\selection_3}}{\tfrelem^s}
    }
  } \\
  \inferrule[MoveRightRestructuringWhole]{
    \wholeSelection{s}{\selection_2} \\
    \performAction{
      \zipper{\pointing{\selection_1}{\selection_3}}{\tfrelem^s_1}
    }{
      \actionlit{move}\ \texttt{right}
    }{
      \zipper{\pointing{\selection_4}{\selection_5}}{\tfrelem^{s'}_2}
    }
  }{
    \performAction{
      \zipper{\restructuring{\selection_1}{\selection_2}{\selection_3}}{\tfrelem^s_1}
    }{
      \actionlit{move}\ \texttt{right}
    }{
      \zipper{\restructuring{\selection_4}{\selection_2}{\selection_5}}{\tfrelem^{s'}_2}
    }
  } \\
  \inferrule[Delete]{
    \fixHolesSelections{\rtipconc{s}}{\filterTiles{s}{\selection_1}}{\filterTiles{s}{\selection_3}}{\ltipconc{s}}{\selection'_1}{\selection'_3}
  }{
    \performAction{
      \zipper{\restructuring{\selection_1}{\selection_2}{\selection_3}}{\tfrelem^s}
    }{
      \actionlit{delete}
    }{
      \zipper{\pointing{\selection'_1}{\selection'_3}}{\tfrelem^s}
    }
  } \\
  \inferrule[MarkRestructuringNotWhole]{
    \text{\note{not\ }}\wholeSelection{s'}{\selection_2} \\
    \fixHolesSelections{\rtipconc{s}}{\selection_1}{\selection_2\selection_3}{\ltipconc{s}}{\selection_4}{\selection_5}
  }{
    \performAction{
      \zipper{\restructuring{\selection_1}{\selection_2}{\selection_3}}{\tfrelem^s}
    }{
      \actionlit{mark}
    }{
      \parseZipper{\zipper{\pointing{\selection_4}{\selection_5}}{\tfrelem^s}}
    }
  } \\
  \inferrule[MarkRestructuringWhole]{
    \wholeSelection{s}{\selection_2} \\
    \fixHolesSelections{\rtipconc{s}}{\selection_1}{\selection_2\selection_3}{\ltipconc{s}}{\selection_4}{\selection_5}
  }{
    \performAction{
      \zipper{\restructuring{\selection_1}{\selection_2}{\selection_3}}{\tfrelem^s}
    }{
      \actionlit{mark}
    }{
      \parseZipper{\zipper{\pointing{\selection_4}{\selection_5}}{\tfrelem^s}}
    }
  }
  \end{mathpar}

  \vspace{-2px}
  \CaptionLabel{Action performing \note{prevent empty hole tile construction, add justification for constructing tiles rather than tokens}}{fig:perform-action}
  \vspace{-2px}
  \end{figure*}

\note{
  Define actions more generally than \tylr~ implementation,
  such that delimiters are assigned sorts to their ends,
  selections are not restricted to shards and
  same-sort tiles, and only selections whose ends are the
  same may enter restructuring mode.
}

% (2 * [3])  ->  (2 * (3 + [_]))

\begin{itemize}
  \item explain action judgment + choice rules
  \item theorems
  \begin{itemize}
    \item movability
    \item selectability
    \item restructuring is sound and complete
  \end{itemize}
\end{itemize}

\begin{theorem}
  Suppose $Z = \zipper{\subject}{\tfrelem^s}$ is a reachable edit state.
  \begin{enumerate}
  \item[(1)] If $\subject = \pointing{\selection_1}{\selection_2}$,
    then $\selection_1\selection_2$ is an opseq of sort $s$.
  \item[(2)] If $\subject = \selecting{\selection_1}{\selection_2}{\selection_3}$,
    then $\parseSelection{\selection_1\selection_2\selection_3}$ is an opseq of sort $s$.
  \item[(3)] If $\subject = \restructuring{\selection_1}{\selection_2}{\selection_3}$...
    \begin{enumerate}
      \item[(a)] ...and $\selection_2$ is intact, then
        $\selection_1\selection_3$ is an opseq of sort $s$.
      \item[(b)] ...and $\selection_2$ is cracked, then
        $\parseSelection{\selection_1\selection_2\selection_3}$ is an opseq of sort $s$.
    \end{enumerate}
  \end{enumerate}
\end{theorem}

\begin{theorem}

\end{theorem}

% \subsection{Typechecking}
% \note{
%   Define tree-structured variants of tiles and frames.
%   Define precedence parser that converts linear forms to tree forms
%   (or assume existence of such parser?).
%   Define type
% }
