% !TEX root = prelim-paper.tex

\section{Tile-based Editor Calculus}\label{sec:formalism}

We now present a precise specification of tile-based
editing in the form of a core calculus called \ty.

Edit states in \ty~ follow a variant
of the zipper pattern first described by Huet.
In particular, \ty~ edit states have a \emph{bottom-up}
zipper structure \note{cite}---consisting of a focused substructure
and, separately, its surrounding ``inside-out'' context---unlike
the top-down structure adopted in other work \note{cite}.
In this work, we call the focused substructure the \emph{subject} of
the zipper and its surrounding context the \emph{frame}.
Sections \note{ref} and \note{ref} present the syntax of
\ty~ subjects and frames as well as methods of disassembling
and reassembling these structures as needed for editing.

The punchline of \ty~ is the action judgment
$\performAction{\editState_1}{\action}{\editState_2}$ \note{section ref},
which sends edit state $\editState_1$ to $\editState_2$
via action $\alpha$.
These actions are governed by a syntactic sensibility theorem \note{ref},
which states that every action-reachable edit state can
be parsed into a language term.
Section \note{ref} presents the action judgment and characterizes
its metatheory.

\subsection{Tiles, tokens, \& selections}
We start with the syntax of the subject of a \ty~ edit state,
shown in Figure \note{ref}.
The subject may be in one of three \emph{modes}.
The first mode, called \emph{pointing mode}, models
a cursor sitting between a pair of \emph{selections},
each of which is a linear sequence of heterogeneously
sorted \emph{tiles} and \emph{tokens}.
We described the syntax of tiles
in the previous section (Figure \note{ref}).
Tokens are the lexical components of tiles, forming either
the substance of childless tiles or the delimiters of parent
tiles' children.
Tokens are generated from tiles via the \emph{tile disassembly} function
$\disassembleTile{\cdot}$, defined in Figure \note{ref},
which takes a tile and produces
a selection consisting of the tile's tokens
and the tile's children tiles.
% Such a permissive syntactic structure is not strictly necessary
% for modeling pointing mode, as \ty ensures that,
% in any action-reachable edit state in pointing mode,
% the selections consist entirely of tiles of the same sort;
% however, it is also useful for modeling intermediate
% edit states internal to the action semantics.

The second mode, called \emph{selecting mode},
consists of a user-selected selection in between
the unselected prefix and suffix selections.
Selecting mode models the user making a selection by dropping
an anchor at the selection's right end and moving
the cursor left to mark the left end.
For brevity, we ignore the mirror case.
% For the sake of brevity, we restrict our attention to selections
% specified by dropping an anchor at the right
% end of the selection and moving left to specify the
% left end; modeling the opposite case yields no new insights.

The third mode, called \emph{restructuring mode},
has the same structure as selecting mode.
It models the user having ``picked up'' a selection
and moving it to different target location within the
prefix and suffix selections.

% Selections may be arbitrarily structured, but we show
% in Section \note{ref} how \ty's actions ensure that,
% in any reachable edit state in pointing mode, the subject
% selections consist entirely of tiles of the same sort.


% The subject may be in one of three
% modes: \emph{pointing}, \emph{selecting}, and \emph{restructuring}.
% Each mode is made up of two or three \emph{selections}, each
% of which is a linear sequence of heterogeneously sorted
% \emph{tiles} and \emph{tokens}.
% Recall that the syntax of tiles was presented in Figure \note{ref}
% of the previous section.
% Tokens are the lexical components of tiles, forming either
% the substance of childless tiles or the delimiters of parent
% tiles' children.
% % Under \ty's actions, they appear in the subject
% % as untethered delimiters of tiles divided by a selection boundary.
% Tokens are generated from tiles via the \emph{tile disassembly} function
% $\disassembleTile{\cdot}$, defined in Figure \note{ref},
% which takes a tile and produces
% a selection consisting of the tile's tokens
% and the tile's children tiles.

Such a disassembly function can be lifted to a disassembly
relation $\stepDisassembleSelection{}{}$ between selections, show in Figure \note{ref}.
Taking t


\begin{figure}
  \vspace{-3px}
  \[
  \arraycolsep=3pt\begin{array}{rlrl}
      \mathsf{Token^{\pat}} & \kappa^{\pat} & ::= &
        \shole ~\vert~
        \svar{x} ~\vert~
        \sprod{}{} ~\vert~
        \texttt{(} ~\vert~
        \texttt{)} \\
      \mathsf{Token^{\expr}} & \kappa^{\expr} & ::= &
        \shole ~\vert~
        \sboollit{b} ~\vert~
        \snumlit{n} ~\vert~
        \svar{x} ~\vert~
        \sprod{}{} ~\vert~
        \splus{}{} ~\vert~
        \smult{}{} \\
      & & \vert &
        \texttt{(} ~\vert~
        \texttt{)} ~\vert~
        \lambda ~\vert~
        \texttt{.} ~\vert~
        \texttt{let} ~\vert~
        \texttt{=} ~\vert~
        \texttt{in}
  \end{array}\]
  \caption{
    Syntax of pattern tokens and expression tokens.
  }
  \label{fig:language-syntax}
\end{figure}
\begin{figure}
  \vspace{-3px}
  \[
  \arraycolsep=3pt\begin{array}{rlrl}
      \mathsf{Selection} & \selection & ::= &
        \selem_1\dots\selem_n \\
      \mathsf{SelectedElement} & \selem & ::= &
        \tile^{\pat} ~\vert~
        \tile^{\expr} ~\vert~
        \shard^{\pat} ~\vert~
        \shard^{\expr}
  \end{array}\]
  \caption{
    Syntax of selections.
  }
  \label{fig:selection-syntax}
\end{figure}
\begin{figure}
  \vspace{-3px}
  \[
  \setlength{\fboxsep}{1pt}
  \arraycolsep=3pt\begin{array}{rcl}
      \disassembleTile{\shole} & = & \tokenLit{\shole} \\
      \disassembleTile{\snumlit{n}} & = & \tokenLit{\snumlit{n}} \\
      \disassembleTile{\sparen{\tile_1\dots\tile_n}}
        & = & \tokenLit{\texttt{(}}\ \tile_1\dots\tile_n\ \tokenLit{\texttt{)}} \\
      & \vdots &
  \end{array}\]
  \caption{
    Disassembling a tile.
  }
  \label{fig:disassemble-tile}
\end{figure}
\begin{figure}
\judgbox{\stepDisassembleSelection{\selection_1}{\selection_2}}{$\selection_1$ step-disassembles down to $\selection_2$}
\begin{mathpar}
  \inferrule[]{
    \disassemblesDown{\selem}{\selection'}
  }{
    \stepDisassembleSelection{\selem\selection}{\selection'\selection}
  }\hspace{40pt}
  \inferrule[]{
    \stepDisassembleSelection{\selection}{\selection'}
  }{
    \stepDisassembleSelection{\selem\selection}{\selem\selection'}
  }
\end{mathpar}
\caption{
  Single-step disassembly of a selection
}
\label{fig:disassemble-selection}
\end{figure}

\note{make clear parallel between CFG derivation and disassembly}

\note{
  discuss the syntax of
  tiles (Fig \ref{fig:tile-syntax}),
  tokens (Fig \ref{fig:token-syntax}),
  selections (Fig \ref{fig:selection-syntax})
}

\note{
  discuss
    disassembly of tiles (Fig \ref{fig:disassemble-tile}),
    disassembly of selections (Fig \ref{fig:disassemble-selection}),
    $\succeq$ partial order and parsing function
}

\begin{definition}
  $\searrow^*$ is the reflexive transitive
  closure of $\searrow$.
\end{definition}

\begin{lemma}
  $\searrow^*$ is a partial order.
\end{lemma}

\begin{lemma}\label{lemma:unique-parsed-selection}
  For every selection $\Psi$, there is a unique maximal
  element $\Psi'$ such that $\Psi'\searrow^*\Psi$.
\end{lemma}

\begin{definition}
  Let $parseSelection$ be the function that takes a selection
  and returns the unique maximal selection guaranteed to exist
  by Lemma \ref{lemma:unique-parsed-selection}.
\end{definition}


\subsection{Frames \& zippers}
\begin{figure}
  \vspace{-3px}
  \[\arraycolsep=3pt\begin{array}{rlrl}
    \text{sequence frame} & \tframe^s & ::= & \tframelit{\tiles^s\framehole\tiles^s}{\tfrelem^s} \\
    \text{pattern tile frame} & \tfrelem^{\pat} & ::= &
      \tframelit{\sparen{\framehole}}{\tframe^{\pat}} \\
    & & \vert &
      \tframelit{\tlam{\framehole}}{\tframe^{\expr}} \\
    & & \vert &
      \tframelit{\tlet{\framehole}{\tiles^{\expr}}}{\tframe^{\expr}} \\
    \text{expression tile frame} & \tfrelem^{\expr} & ::= &
      \froot \\
    & & \vert &
      \tframelit{\sparen{\framehole}}{\tframe^{\expr}} \\
    & & \vert &
      \tframelit{\tlet{\tiles^{\pat}}{\framehole}}{\tframe^{\expr}}
      % \scond{}{\tiles^{\expr}}{}
  \end{array}\]
  \caption{
    Syntax of pattern and expression frames
  }
  \label{fig:frame-syntax}
\end{figure}

\begin{figure}
  \vspace{-3px}
  \[
  \arraycolsep=3pt\begin{array}{rlrl}
      \mathsf{EditState} & \editState & ::= & \zip^{\pat} ~\vert~ \zip^{\expr} \\
      \mathsf{Zipper} & \zip^s & ::= & \zipper{\subject}{\tfrelem^s} \\
      % \mathsf{Tile}^{\typ} & \tile^{\typ} & ::= &
      %     % \tnum ~\vert~
      %     \shole ~\vert~
      %     \sbool ~\vert~
      %     \snum ~\vert~
      %     \sarr{}{} ~\vert~
      %     \sprod{}{} ~\vert~
      %     \sparen{\tiles^{\typ}}\\
      \mathsf{Subject} & \subject & ::= &
        \pointing{\selection}{\selection} ~\vert~
        \selecting{\selection}{\selection}{\selection} ~\vert~
        \restructuring{\selection}{\selection}{\selection}
  \end{array}\]
  \caption{
    Syntax of zippers \note{find different notation for selections/restructuring, explain directionality/minimality}
  }
  \label{fig:zipper-syntax}
\end{figure}

\begin{figure}
  \vspace{-3px}
  \[
  \setlength{\fboxsep}{1pt}
  \arraycolsep=3pt\def\arraystretch{1.4}\begin{array}{rcl}
      \disassembleTileFrame{
        \tframelit{
          \sparen{\_}
        }{
          \tframelit{\tiles^s_1\_\tiles^s_2}{\tfrelem^{\pat}}
        }
      } & = &
        \zipper{
          \pointing{\tiles^s_1\tokenLit{\texttt{(}}~}{~\tokenLit{\texttt{)}}\tiles^s_2}
        }{
          \tfrelem^{\pat}
        } \\

      \disassembleTileFrame{
        \tframelit{
          \slam{\_}{}
        }{
          \tframelit{\tiles^{\expr}_1\_\tiles^{\expr}_2}{\tfrelem^{\expr}}
        }
      } & = &
        \zipper{
          \pointing{\tiles^{\expr}_1\tokenLit{\lambda}~}{~\tokenLit{\texttt{.}}\tiles^{\expr}_2}
        }{
          \tfrelem^{\expr}
        } \\

      \disassembleTileFrame{
        \tframelit{
          \slet{\_}{\tiles^{\expr}_0}{}
        }{
          \tframelit{\tiles^{\expr}_1\_\tiles^{\expr}_2}{\tfrelem^{\expr}}
        }
      } & = & \\
        \zipper{
          \pointing{
            \tiles^{\expr}_1\tokenLit{\texttt{let}}~
          }{
            ~\tokenLit{\texttt{=}}~\tiles^{\expr}_0\tokenLit{\texttt{in}}~\tiles^{\expr}_2
          }
        }{
          \tfrelem^{\expr}
        } \\

      \disassembleTileFrame{
        \tframelit{
          \slet{\tiles^{\pat}}{\_}{}
        }{
          \tframelit{\tiles^{\expr}_1\_\tiles^{\expr}_2}{\tfrelem^{\expr}}
        }
      } & = & \\
        \zipper{
          \pointing{
            \tiles^{\expr}_1\tokenLit{\texttt{let}}~\tiles^{\pat}~\tokenLit{\texttt{=}}~
          }{
            ~\tokenLit{\texttt{in}}~\tiles^{\expr}_2
          }
        }{
          \tfrelem^{\expr}
        }
  \end{array}\]
  \caption{
    Disassembling a tile frame. \note{fix spacing}
  }
  \label{fig:disassemble-tile}
\end{figure}

\note{give an example of a frame to clarify what's going on}

\begin{definition}
  $\zeta^s_1\nearrow\zeta^s_2$ if $\disassembleTileFrame{\zeta^s_1} = \zeta^s_2$.
  \note{defined this for symmetry with diassembling of tiles}
\end{definition}

\begin{definition}
  $\nearrow^*$ is the reflexive transitive closure of $\nearrow$.
\end{definition}

\begin{lemma}
  $\nearrow^*$ is a partial order.
\end{lemma}

\begin{lemma}
  For every zipper $\zip^s_1 = \zipper{\pointing{\selection_1}{\selection_2}}{\tfrelem^s}$, there exists
  a unique minimal element $\zip^s_2$ such that $\zip^s_2 \nearrow^* \zip^s_1$.
\end{lemma}

\begin{definition}
  Let $parseZipper$ be the function that takes a pointing mode
  zipper
  and returns the unique minimal pointing mode zipper guaranteed to exist
  by Lemma \ref{lemma:unique-parsed-selection}.
\end{definition}

\begin{figure}
  \vspace{-3px}
  \[
  \arraycolsep=3pt\begin{array}{rlrl}
      \text{tip} & \tip & ::= & \lltip{s} ~\vert~ \rrtip{s}
  \end{array}\]

  \judgbox{
    \text{$L$ is $\tau_1\tau_2$-connected}
  }{}
  \begin{mathpar}
    \inferrule[]{
      \leftTip{\selem} = \tau_1 \\
      \rightTip{\selem} = \tau_2
    }{
      \text{$\selem$ is $\tau_1\tau_2$-connected}
    }

    \inferrule[]{
      \text{$\selection_1$ is $\tau_1\tau_2$-connected} \\
      \text{$\selection_2$ is $\tau_2\tau_3$-connected}
    }{
      \text{$\selection_1\selection_2$ is $\tau_1\tau_3$-connected}
    }
  \end{mathpar}

  \caption{
    Syntax of tile and token tips
  }
  \label{fig:tip-syntax}
\end{figure}

% \begin{figure}
  \newcommand{\spacing}{\ \ \ \ \ }
  \[
  \setlength{\fboxsep}{1pt}
  \arraycolsep=3pt\def\arraystretch{1.25}\begin{array}{lcll}
    % \fixHolesFn{\selection_1}{\selection_2} & = &
    %   \begin{cases}
    %     (~\cdot~, \ophole\selection_2) & \text{if } \selection_1 =~\cdot~ \text{ and } \leftTip{\selection_2} =\ \rtip
    %   \end{cases} \\
      \fixHolesFn{~\cdot~}{\selem\selection} & = &
        (~\cdot~, \ophole\selem\selection) & \text{\spacing if } \leftTip{\selem} =\ \rtip \\
     \fixHolesFn{\selection\selem}{~\cdot~} & = &
       (\selection\selem\ophole, ~\cdot~) & \text{\spacing if } \rightTip{\selem} = \ltip \\
    \fixHolesFn{\selection\ophole}{\selem\selection'} & = &
      (\selection, \selem\selection') & \text{\spacing if } \leftTip{\selem} = \ltip \\
   \fixHolesFn{\selection\selem}{\ophole\selection'} & = &
     (\selection\selem, \selection') & \text{\spacing if } \rightTip{\selem} =\ \rtip \\
  \fixHolesFn{\selection\binhole}{\selem\selection'} & = &
      (\selection, \selem\selection') & \text{\spacing if } \leftTip{\selem} =\ \rtip \\
  \fixHolesFn{\selection\selem}{\binhole\selection'} & = &
    (\selection\selem, \selection') & \text{\spacing if } \rightTip{\selem} = \ltip \\
  \fixHolesFn{\selection\selem}{\selem'\selection'} & = &
      (\selection\selem, \ophole\selem'\selection') & \text{\spacing if } \rightTip{\selem} = \ltip \text{ and }\leftTip{\selem'} =\ \rtip \\
  \fixHolesFn{\selection\selem}{\selem'\selection'} & = &
    (\selection\selem, \binhole\selem'\selection') & \text{\spacing if } \rightTip{\selem} =\ \rtip \text{ and }\leftTip{\selem'} = \ltip \\
  \fixHolesFn{\selection}{\selection'} & = & (\selection, \selection') & \text{\spacing otherwise}
\end{array}\]
  \vspace{-2px}
  \CaptionLabel{Hole fixing}{fig:hole-fixing}
  \vspace{-2px}
  \end{figure}
\begin{figure}
  % \judgbox{
  %   \fixholes{\tip_1}{\selection}{\tip_2}{\selection'}
  % }{
  %   $\selection$ is hole-fixed to $\tip_1\tip_2$-connected $\selection'$
  % }
  % \begin{mathpar}
  %   \inferrule[]{
  %   }{
  %     \fixholes{\lltip{s}}{\cdot}{\rrtip{s}}{\optile{\ophole}^s}
  %   }\hspace{20pt}
  %   \inferrule[]{
  %   }{
  %     \fixholes{\rrtip{s}}{\cdot}{\lltip{s}}{\bintile{\binhole}^s}
  %   }

  %   \inferrule[]{
  %     \fixholes{\tau_1}{\selection}{\tau_2}{\selection'}
  %   }{
  %     \fixholes{\tau_1}{\optile{\ophole}\selection}{\tau_2}{\selection'}
  %   }\hspace{20pt}
  %   \inferrule[]{
  %     \fixholes{\tau_1}{\selection}{\tau_2}{\selection'}
  %   }{
  %     \fixholes{\tau_1}{\bintile{\binhole}\selection}{\tau_2}{\selection'}
  %   }

  %   \inferrule[]{
  %     \leftTip{\selem} = \tau_1 \\
  %     \fixholes{\rightTip{\selem}}{\selection}{\tau_2}{\selection'}
  %   }{
  %     \fixholes{\tau_1}{\selem\selection}{\tau_2}{\selem\selection'}
  %   }
  % \end{mathpar}

  \judgbox{
    \fixHolesSelection{\tip_1}{\selection_1}{\selection_2}{\tip_2}
  }{
    $\selection_1$ is \note{empty} hole fixed under tip constraint $\tip_1$ \\
    to produce $\selection_2$ and new tip constraint $\tip_2$
  }
  \begin{mathpar}
    \inferrule[]{
    }{
      \fixHolesSelection{\tip}{\cdot}{\cdot}{\tip}
    } \\
    \inferrule[]{
      \text{\note{$\selem$ is a hole}} \\
      \fixHolesSelection{\tip}{\selection}{\selection'}{\tip'}
    }{
      \fixHolesSelection{\tip}{\selem\selection}{\selection'}{\tip'}
    } \\
    \inferrule[]{
      \text{\note{$\selem$ not a hole}} \\
      \fits{\tip}{\leftTip{\selem}} \\
      \fixHolesSelection{\rightTip{\selem}}{\selection}{\selection'}{\tip'}
    }{
      \fixHolesSelection{r}{\selem\selection}{\selection'}{\tip'}
    } \\
    \inferrule[]{
      \text{\note{$\selem$ not a hole}} \\
      \leftTip{\selem} = \ltipconc{s} \\
      \fixHolesSelection{\rightTip{\selem}}{\selection}{\selection'}{\tip'}
    }{
      \fixHolesSelection{\lltip{s}}{\selem\selection}{\optile{\ophole}^s\selem\selection'}{\rTip'}
    } \\
    \inferrule[]{
      \text{\note{$\selem$ not a hole}} \\
      \leftTip{\selem} = \ltipconv{s} \\
      \fixHolesSelection{\rightTip{\selem}}{\selection}{\selection'}{\rTip'}
    }{
      \fixHolesSelection{\rrtip{s}}{\selem\selection}{\bintile{\binhole}^s\selem\selection'}{\rTip'}
    } \\
  \end{mathpar}

  \judgbox{
    \fixHolesSelections{\selection_1}{\selection_2}{\selection_3}{\selection_4}
  }{$\selection_1$ and $\selection_2$ are hole fixed \\
    \ \ to produce $\selection_3$ and $\selection_4$
  }
  \vspace{10pt}
  \begin{mathpar}
    \inferrule[]{
      \fixHolesSelection{\lltip{s}}{\selection_1}{\selection'_1}{\tip} \\
      \fixHolesSelection{\tip}{\selection_2}{\selection'_2}{\rrtip{s}}
    }{
      \fixHolesSelections{\selection_1}{\selection_2}{\selection'_1}{\selection'_2}
    } \\
    \inferrule[]{
      \fixHolesSelection{\lltip{s}}{\selection_1}{\selection'_1}{\tip} \\
      \fixHolesSelection{\tip}{\selection_2}{\selection'_2}{\lltip{s}}
    }{
      \fixHolesSelections{\selection_1}{\selection_2}{\selection'_1}{\selection'_2\optile{\ophole}^s}
    }
  \end{mathpar}
  \caption{
    Hole fixing \note{add sort subscript to judgment form}
  }
  \label{fig:fixholes-2}
  \end{figure}
\begin{figure}
  \newcommand{\spacing}{\ \ \ \ \ }
  \[
  \setlength{\fboxsep}{1pt}
  \arraycolsep=3pt\def\arraystretch{1.25}\begin{array}{lcl}
    % \fixHolesFn{\selection_1}{\selection_2} & = &
    %   \begin{cases}
    %     (~\cdot~, \ophole\selection_2) & \text{if } \selection_1 =~\cdot~ \text{ and } \leftTip{\selection_2} =\ \rtip
    %   \end{cases} \\
      \filterTiles{s}{\cdot} & = & \cdot \\
      \filterTiles{s}{\shard^{s'}\selection} & = & \filterTiles{s}{\selection} \\
      \filterTiles{s}{\tile^{s'}\selection} & = &
        \begin{cases}
          \tile^{s'}\filterTiles{s}{\selection} & \text{ if } s = s' \\
          \filterTiles{s}{\selection} & \text{ else }
        \end{cases}
\end{array}\]
  \vspace{-2px}
  \CaptionLabel{Filtering tiles}{fig:filter-tiles}
  \vspace{-2px}
  \end{figure}
\begin{figure}
  \vspace{-3px}
  \judgbox{
    \wholeSelection{s}{\selection}
  }{
    $\selection$ consists of tiles of sort $s$
  }
  \begin{mathpar}
    \inferrule[]{
    }{
      \wholeSelection{s}{\cdot}
    } \ \ \ \ \ \ \ \ \ \
    \inferrule[]{
      \wholeSelection{s}{\selection}
    }{
      \wholeSelection{s}{\tile^s\selection}
    }
  \end{mathpar}
  \caption{
    Whole selections
  }
  \label{fig:whole-selection}
\end{figure}

\begin{figure}
  \vspace{-3px}
  \[
  \arraycolsep=3pt\begin{array}{rlrl}
      \text{action} & \action & ::= &
        \actionlit{mark} ~\vert~
        \actionlit{move}~\direction ~\vert~
        \actionlit{insert}~\tile ~\vert~
        \actionlit{remove} \\
      \text{direction} & \direction & ::= &
        \texttt{left} ~\vert~
        \texttt{right}
  \end{array}\]
  \caption{Syntax of actions}
  \label{fig:action-syntax}
\end{figure}

\begin{figure}
  \judgbox{\performAction{\editState_1}{\action}{\editState_2}}{$\editState_1$ transitions via $\alpha$ to $\editState_2$}
  \begin{mathpar}
  \inferrule[MoveRightEnterTile]{
    \disassembleTile{\tile^s} = \token^{s}\selection\ \ \ \ \
    parseZipper(
      \zipper{\pointing{\tiles^s_1\token^{s}~}{~\selection\tiles^s_2}}{\tframe^s}
    ) = \zeta^{s'}
  }{
    \performAction{
      \zipper{\pointing{\tiles^s_1}{\tile^s\tiles^s_2}}{\tframe^s}
    }{
      \actionlit{move}\ \texttt{right}
    }{
      \zeta^{s'}
    }
  } \\
  \inferrule[MoveRightExitTile]{
    \disassembleTileFrame{\zipper{\pointing{\tiles^s}{\cdot}}{\tfrelem^s}} =
    \zipper{\pointing{\selection_1}{\token^{s'}\selection_2}}{\tfrelem^{s'}}\ \ \ \ \
    parseZipper(
      \zipper{\pointing{\selection_1\token^{s'}}{\selection_2}}{\tfrelem^{s'}}
    ) = \zeta^{s''}
  }{
    \performAction{
      \zipper{\pointing{\tiles^s}{\cdot}}{\tfrelem^s}
    }{
      \actionlit{move}\ \texttt{right}
    }{
      \zeta^{s''}
    }
  }
  \end{mathpar}

  \vspace{-2px}
  \CaptionLabel{Bidirectional Typing of External Expressions}{fig:bidirectional-typing}
  \vspace{-2px}
  \end{figure}

\subsubsection{Actions}
\note{
  Define actions more generally than \tylr~ implementation,
  such that delimiters are assigned sorts to their ends,
  selections are not restricted to shards and
  same-sort tiles, and only selections whose ends are the
  same may enter restructuring mode.
}

% (2 * [3])  ->  (2 * (3 + [_]))

\begin{itemize}
  \item explain action judgment + choice rules
  \item theorems
  \begin{itemize}
    \item movability
    \item selectability
    \item restructuring is sound and complete
  \end{itemize}
\end{itemize}

\subsection{Typechecking}
\note{
  Define tree-structured variants of tiles and frames.
  Define precedence parser that converts linear forms to tree forms
  (or assume existence of such parser?).
  Define type
}
