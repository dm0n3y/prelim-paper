% !TEX root = prelim-paper.tex

\section{Editor Calculus}\label{sec:formalism}

\note{think about name of calculus as opposed to implementation}

\subsection{Tiles, tokens, \& selections}
\begin{figure}
  \vspace{-3px}
  \[
  \arraycolsep=3pt\begin{array}{rlrl}
      \mathsf{Token^{\pat}} & \kappa^{\pat} & ::= &
        \shole ~\vert~
        \svar{x} ~\vert~
        \sprod{}{} ~\vert~
        \texttt{(} ~\vert~
        \texttt{)} \\
      \mathsf{Token^{\expr}} & \kappa^{\expr} & ::= &
        \shole ~\vert~
        \sboollit{b} ~\vert~
        \snumlit{n} ~\vert~
        \svar{x} ~\vert~
        \sprod{}{} ~\vert~
        \splus{}{} ~\vert~
        \smult{}{} \\
      & & \vert &
        \texttt{(} ~\vert~
        \texttt{)} ~\vert~
        \lambda ~\vert~
        \texttt{.} ~\vert~
        \texttt{let} ~\vert~
        \texttt{=} ~\vert~
        \texttt{in}
  \end{array}\]
  \caption{
    Syntax of pattern tokens and expression tokens.
  }
  \label{fig:language-syntax}
\end{figure}
\begin{figure}
  \vspace{-3px}
  \[
  \arraycolsep=3pt\begin{array}{rlrl}
      \mathsf{Selection} & \selection & ::= &
        \selem_1\dots\selem_n \\
      \mathsf{SelectedElement} & \selem & ::= &
        \tile^{\pat} ~\vert~
        \tile^{\expr} ~\vert~
        \shard^{\pat} ~\vert~
        \shard^{\expr}
  \end{array}\]
  \caption{
    Syntax of selections.
  }
  \label{fig:selection-syntax}
\end{figure}

\note{refer to Figures \ref{fig:language-syntax} \& \ref{fig:tile-syntax}}


\subsection{Frames}

\subsection{Edit States}
% \begin{figure}
  \vspace{-3px}
  \[
  \arraycolsep=3pt\begin{array}{rlrl}
      \mathsf{Token^{\pat}} & \kappa^{\pat} & ::= &
        \shole ~\vert~
        \svar{x} ~\vert~
        \sprod{}{} ~\vert~
        \texttt{(} ~\vert~
        \texttt{)} \\
      \mathsf{Token^{\expr}} & \kappa^{\expr} & ::= &
        \shole ~\vert~
        \sboollit{b} ~\vert~
        \snumlit{n} ~\vert~
        \svar{x} ~\vert~
        \sprod{}{} ~\vert~
        \splus{}{} ~\vert~
        \smult{}{} \\
      & & \vert &
        \texttt{(} ~\vert~
        \texttt{)} ~\vert~
        \lambda ~\vert~
        \texttt{.} ~\vert~
        \texttt{let} ~\vert~
        \texttt{=} ~\vert~
        \texttt{in}
  \end{array}\]
  \caption{
    Syntax of pattern tokens and expression tokens.
  }
  \label{fig:language-syntax}
\end{figure}

\subsubsection{Actions}
\note{
  Define actions more generally than \tylr~ implementation,
  such that delimiters are assigned sorts to their ends,
  selections are not restricted to shards and
  same-sort tiles, and only selections whose ends are the
  same may enter restructuring mode.
}

% (2 * [3])  ->  (2 * (3 + [_]))

\begin{itemize}
  \item explain action judgment + choice rules
  \item theorems
  \begin{itemize}
    \item movability
    \item selectability
    \item restructuring is sound and complete
  \end{itemize}
\end{itemize}

\subsection{Typechecking}
\note{
  Define tree-structured variants of tiles and frames.
  Define precedence parser that converts linear forms to tree forms
  (or assume existence of such parser?).
  Define type
}
