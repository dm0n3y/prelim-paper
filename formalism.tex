% !TEX root = prelim-paper.tex

\section{Editor Calculus}\label{sec:formalism}

\note{think about name of calculus as opposed to implementation}

\subsection{Tiles, tokens, \& selections}
\begin{figure}
  \vspace{-3px}
  \[
  \arraycolsep=3pt\begin{array}{rlrl}
      \mathsf{Token^{\pat}} & \kappa^{\pat} & ::= &
        \shole ~\vert~
        \svar{x} ~\vert~
        \sprod{}{} ~\vert~
        \texttt{(} ~\vert~
        \texttt{)} \\
      \mathsf{Token^{\expr}} & \kappa^{\expr} & ::= &
        \shole ~\vert~
        \sboollit{b} ~\vert~
        \snumlit{n} ~\vert~
        \svar{x} ~\vert~
        \sprod{}{} ~\vert~
        \splus{}{} ~\vert~
        \smult{}{} \\
      & & \vert &
        \texttt{(} ~\vert~
        \texttt{)} ~\vert~
        \lambda ~\vert~
        \texttt{.} ~\vert~
        \texttt{let} ~\vert~
        \texttt{=} ~\vert~
        \texttt{in}
  \end{array}\]
  \caption{
    Syntax of pattern tokens and expression tokens.
  }
  \label{fig:language-syntax}
\end{figure}
\begin{figure}
  \vspace{-3px}
  \[
  \arraycolsep=3pt\begin{array}{rlrl}
      \mathsf{Selection} & \selection & ::= &
        \selem_1\dots\selem_n \\
      \mathsf{SelectedElement} & \selem & ::= &
        \tile^{\pat} ~\vert~
        \tile^{\expr} ~\vert~
        \shard^{\pat} ~\vert~
        \shard^{\expr}
  \end{array}\]
  \caption{
    Syntax of selections.
  }
  \label{fig:selection-syntax}
\end{figure}
\begin{figure}
  \vspace{-3px}
  \[
  \setlength{\fboxsep}{1pt}
  \arraycolsep=3pt\begin{array}{rcl}
      \disassembleTile{\shole} & = & \tokenLit{\shole} \\
      \disassembleTile{\snumlit{n}} & = & \tokenLit{\snumlit{n}} \\
      \disassembleTile{\sparen{\tile_1\dots\tile_n}}
        & = & \tokenLit{\texttt{(}}\ \tile_1\dots\tile_n\ \tokenLit{\texttt{)}} \\
      & \vdots &
  \end{array}\]
  \caption{
    Disassembling a tile.
  }
  \label{fig:disassemble-tile}
\end{figure}
\begin{figure}
\judgbox{\stepDisassembleSelection{\selection_1}{\selection_2}}{$\selection_1$ step-disassembles down to $\selection_2$}
\begin{mathpar}
  \inferrule[]{
    \disassemblesDown{\selem}{\selection'}
  }{
    \stepDisassembleSelection{\selem\selection}{\selection'\selection}
  }\hspace{40pt}
  \inferrule[]{
    \stepDisassembleSelection{\selection}{\selection'}
  }{
    \stepDisassembleSelection{\selem\selection}{\selem\selection'}
  }
\end{mathpar}
\caption{
  Single-step disassembly of a selection
}
\label{fig:disassemble-selection}
\end{figure}

\note{
  discuss the syntax of
  tiles (Fig \ref{fig:tile-syntax}),
  tokens (Fig \ref{fig:token-syntax}),
  selections (Fig \ref{fig:selection-syntax})
}

\note{
  discuss
    disassembly of tiles (Fig \ref{fig:disassemble-tile}),
    disassembly of selections (Fig \ref{fig:disassemble-selection}),
    $\succeq$ partial order and parsing function
}

\begin{definition}
  $\searrow^*$ is the reflexive transitive
  closure of $\searrow$.
\end{definition}

\begin{lemma}
  $\searrow^*$ is a partial order.
\end{lemma}

\begin{lemma}\label{lemma:unique-parsed-selection}
  For every selection $\Psi$, there is a unique maximal
  element $\Psi'$ such that $\Psi'\searrow^*\Psi$.
\end{lemma}

\begin{definition}
  Let $parseSelection$ be the function that takes a selection
  and returns the unique maximal selection guaranteed to exist
  by Lemma \ref{lemma:unique-parsed-selection}.
\end{definition}


\subsection{Frames \& zippers}
\begin{figure}
  \vspace{-3px}
  \[\arraycolsep=3pt\begin{array}{rlrl}
    \text{sequence frame} & \tframe^s & ::= & \tframelit{\tiles^s\framehole\tiles^s}{\tfrelem^s} \\
    \text{pattern tile frame} & \tfrelem^{\pat} & ::= &
      \tframelit{\sparen{\framehole}}{\tframe^{\pat}} \\
    & & \vert &
      \tframelit{\tlam{\framehole}}{\tframe^{\expr}} \\
    & & \vert &
      \tframelit{\tlet{\framehole}{\tiles^{\expr}}}{\tframe^{\expr}} \\
    \text{expression tile frame} & \tfrelem^{\expr} & ::= &
      \froot \\
    & & \vert &
      \tframelit{\sparen{\framehole}}{\tframe^{\expr}} \\
    & & \vert &
      \tframelit{\tlet{\tiles^{\pat}}{\framehole}}{\tframe^{\expr}}
      % \scond{}{\tiles^{\expr}}{}
  \end{array}\]
  \caption{
    Syntax of pattern and expression frames
  }
  \label{fig:frame-syntax}
\end{figure}

\begin{figure}
  \vspace{-3px}
  \[
  \arraycolsep=3pt\begin{array}{rlrl}
      \mathsf{EditState} & \editState & ::= & \zip^{\pat} ~\vert~ \zip^{\expr} \\
      \mathsf{Zipper} & \zip^s & ::= & \zipper{\subject}{\tfrelem^s} \\
      % \mathsf{Tile}^{\typ} & \tile^{\typ} & ::= &
      %     % \tnum ~\vert~
      %     \shole ~\vert~
      %     \sbool ~\vert~
      %     \snum ~\vert~
      %     \sarr{}{} ~\vert~
      %     \sprod{}{} ~\vert~
      %     \sparen{\tiles^{\typ}}\\
      \mathsf{Subject} & \subject & ::= &
        \pointing{\selection}{\selection} ~\vert~
        \selecting{\selection}{\selection}{\selection} ~\vert~
        \restructuring{\selection}{\selection}{\selection}
  \end{array}\]
  \caption{
    Syntax of zippers \note{find different notation for selections/restructuring, explain directionality/minimality}
  }
  \label{fig:zipper-syntax}
\end{figure}

\begin{figure}
  \vspace{-3px}
  \[
  \setlength{\fboxsep}{1pt}
  \arraycolsep=3pt\def\arraystretch{1.4}\begin{array}{rcl}
      \disassembleTileFrame{
        \tframelit{
          \sparen{\_}
        }{
          \tframelit{\tiles^s_1\_\tiles^s_2}{\tfrelem^{\pat}}
        }
      } & = &
        \zipper{
          \pointing{\tiles^s_1\tokenLit{\texttt{(}}~}{~\tokenLit{\texttt{)}}\tiles^s_2}
        }{
          \tfrelem^{\pat}
        } \\

      \disassembleTileFrame{
        \tframelit{
          \slam{\_}{}
        }{
          \tframelit{\tiles^{\expr}_1\_\tiles^{\expr}_2}{\tfrelem^{\expr}}
        }
      } & = &
        \zipper{
          \pointing{\tiles^{\expr}_1\tokenLit{\lambda}~}{~\tokenLit{\texttt{.}}\tiles^{\expr}_2}
        }{
          \tfrelem^{\expr}
        } \\

      \disassembleTileFrame{
        \tframelit{
          \slet{\_}{\tiles^{\expr}_0}{}
        }{
          \tframelit{\tiles^{\expr}_1\_\tiles^{\expr}_2}{\tfrelem^{\expr}}
        }
      } & = & \\
        \zipper{
          \pointing{
            \tiles^{\expr}_1\tokenLit{\texttt{let}}~
          }{
            ~\tokenLit{\texttt{=}}~\tiles^{\expr}_0\tokenLit{\texttt{in}}~\tiles^{\expr}_2
          }
        }{
          \tfrelem^{\expr}
        } \\

      \disassembleTileFrame{
        \tframelit{
          \slet{\tiles^{\pat}}{\_}{}
        }{
          \tframelit{\tiles^{\expr}_1\_\tiles^{\expr}_2}{\tfrelem^{\expr}}
        }
      } & = & \\
        \zipper{
          \pointing{
            \tiles^{\expr}_1\tokenLit{\texttt{let}}~\tiles^{\pat}~\tokenLit{\texttt{=}}~
          }{
            ~\tokenLit{\texttt{in}}~\tiles^{\expr}_2
          }
        }{
          \tfrelem^{\expr}
        }
  \end{array}\]
  \caption{
    Disassembling a tile frame. \note{fix spacing}
  }
  \label{fig:disassemble-tile}
\end{figure}

\note{give an example of a frame to clarify what's going on}

\begin{definition}
  $\zeta^s_1\nearrow\zeta^s_2$ if $\disassembleTileFrame{\zeta^s_1} = \zeta^s_2$.
  \note{defined this for symmetry with diassembling of tiles}
\end{definition}

\begin{definition}
  $\nearrow^*$ is the reflexive transitive closure of $\nearrow$.
\end{definition}

\begin{lemma}
  $\nearrow^*$ is a partial order.
\end{lemma}

\begin{lemma}
  For every zipper $\zip^s_1 = \zipper{\pointing{\selection_1}{\selection_2}}{\tfrelem^s}$, there exists
  a unique minimal element $\zip^s_2$ such that $\zip^s_2 \nearrow^* \zip^s_1$.
\end{lemma}

\begin{definition}
  Let $parseZipper$ be the function that takes a selection
  and returns the unique maximal selection guaranteed to exist
  by Lemma \ref{lemma:unique-parsed-selection}.
\end{definition}


\begin{figure}
  \vspace{-3px}
  \[
  \arraycolsep=3pt\begin{array}{rlrl}
      \text{action} & \action & ::= &
        \actionlit{mark} ~\vert~
        \actionlit{move}~\direction ~\vert~
        \actionlit{insert}~\tile ~\vert~
        \actionlit{remove} \\
      \text{direction} & \direction & ::= &
        \texttt{left} ~\vert~
        \texttt{right}
  \end{array}\]
  \caption{Syntax of actions}
  \label{fig:action-syntax}
\end{figure}

\begin{figure}
  \judgbox{\performAction{\editState_1}{\action}{\editState_2}}{$\editState_1$ transitions via $\alpha$ to $\editState_2$}
  \begin{mathpar}
  \inferrule[MoveRightEnterTile]{
    \disassembleTile{\tile^s} = \token^{s}\selection\ \ \ \ \
    parseZipper(
      \zipper{\pointing{\tiles^s_1\token^{s}~}{~\selection\tiles^s_2}}{\tframe^s}
    ) = \zeta^{s'}
  }{
    \performAction{
      \zipper{\pointing{\tiles^s_1}{\tile^s\tiles^s_2}}{\tframe^s}
    }{
      \actionlit{move}\ \texttt{right}
    }{
      \zeta^{s'}
    }
  } \\
  \inferrule[MoveRightExitTile]{
    \disassembleTileFrame{\zipper{\pointing{\tiles^s}{\cdot}}{\tfrelem^s}} =
    \zipper{\pointing{\selection_1}{\token^{s'}\selection_2}}{\tfrelem^{s'}}\ \ \ \ \
    parseZipper(
      \zipper{\pointing{\selection_1\token^{s'}}{\selection_2}}{\tfrelem^{s'}}
    ) = \zeta^{s''}
  }{
    \performAction{
      \zipper{\pointing{\tiles^s}{\cdot}}{\tfrelem^s}
    }{
      \actionlit{move}\ \texttt{right}
    }{
      \zeta^{s''}
    }
  }
  \end{mathpar}

  \vspace{-2px}
  \CaptionLabel{Bidirectional Typing of External Expressions}{fig:bidirectional-typing}
  \vspace{-2px}
  \end{figure}

\subsubsection{Actions}
\note{
  Define actions more generally than \tylr~ implementation,
  such that delimiters are assigned sorts to their ends,
  selections are not restricted to shards and
  same-sort tiles, and only selections whose ends are the
  same may enter restructuring mode.
}

% (2 * [3])  ->  (2 * (3 + [_]))

\begin{itemize}
  \item explain action judgment + choice rules
  \item theorems
  \begin{itemize}
    \item movability
    \item selectability
    \item restructuring is sound and complete
  \end{itemize}
\end{itemize}

\subsection{Typechecking}
\note{
  Define tree-structured variants of tiles and frames.
  Define precedence parser that converts linear forms to tree forms
  (or assume existence of such parser?).
  Define type
}
