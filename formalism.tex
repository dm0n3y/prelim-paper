% !TEX root = prelim-paper.tex

\section{Editor Calculus}\label{sec:formalism}

\note{think about name of calculus as opposed to implementation}

\subsection{Tiles, tokens, \& selections}
\begin{figure}
  \vspace{-3px}
  \[
  \arraycolsep=3pt\begin{array}{rlrl}
      \mathsf{Token^{\pat}} & \kappa^{\pat} & ::= &
        \shole ~\vert~
        \svar{x} ~\vert~
        \sprod{}{} ~\vert~
        \texttt{(} ~\vert~
        \texttt{)} \\
      \mathsf{Token^{\expr}} & \kappa^{\expr} & ::= &
        \shole ~\vert~
        \sboollit{b} ~\vert~
        \snumlit{n} ~\vert~
        \svar{x} ~\vert~
        \sprod{}{} ~\vert~
        \splus{}{} ~\vert~
        \smult{}{} \\
      & & \vert &
        \texttt{(} ~\vert~
        \texttt{)} ~\vert~
        \lambda ~\vert~
        \texttt{.} ~\vert~
        \texttt{let} ~\vert~
        \texttt{=} ~\vert~
        \texttt{in}
  \end{array}\]
  \caption{
    Syntax of pattern tokens and expression tokens.
  }
  \label{fig:language-syntax}
\end{figure}
\begin{figure}
  \vspace{-3px}
  \[
  \arraycolsep=3pt\begin{array}{rlrl}
      \mathsf{Selection} & \selection & ::= &
        \selem_1\dots\selem_n \\
      \mathsf{SelectedElement} & \selem & ::= &
        \tile^{\pat} ~\vert~
        \tile^{\expr} ~\vert~
        \shard^{\pat} ~\vert~
        \shard^{\expr}
  \end{array}\]
  \caption{
    Syntax of selections.
  }
  \label{fig:selection-syntax}
\end{figure}
\begin{figure}
  \vspace{-3px}
  \[
  \setlength{\fboxsep}{1pt}
  \arraycolsep=3pt\begin{array}{rcl}
      \disassembleTile{\shole} & = & \tokenLit{\shole} \\
      \disassembleTile{\snumlit{n}} & = & \tokenLit{\snumlit{n}} \\
      \disassembleTile{\sparen{\tile_1\dots\tile_n}}
        & = & \tokenLit{\texttt{(}}\ \tile_1\dots\tile_n\ \tokenLit{\texttt{)}} \\
      & \vdots &
  \end{array}\]
  \caption{
    Disassembling a tile.
  }
  \label{fig:disassemble-tile}
\end{figure}
\begin{figure}
\judgbox{\stepDisassembleSelection{\selection_1}{\selection_2}}{$\selection_1$ step-disassembles down to $\selection_2$}
\begin{mathpar}
  \inferrule[]{
    \disassemblesDown{\selem}{\selection'}
  }{
    \stepDisassembleSelection{\selem\selection}{\selection'\selection}
  }\hspace{40pt}
  \inferrule[]{
    \stepDisassembleSelection{\selection}{\selection'}
  }{
    \stepDisassembleSelection{\selem\selection}{\selem\selection'}
  }
\end{mathpar}
\caption{
  Single-step disassembly of a selection
}
\label{fig:disassemble-selection}
\end{figure}

\note{
  discuss the syntax of
  tiles (Fig \ref{fig:tile-syntax}),
  tokens (Fig \ref{fig:token-syntax}),
  selections (Fig \ref{fig:selection-syntax})
}

\note{
  discuss
    disassembly of tiles (Fig \ref{fig:disassemble-tile}),
    disassembly of selections (Fig \ref{fig:disassemble-selection}),
    $\succeq$ partial order and parsing function
}

\begin{definition}
  $\disassembleSelection{\Psi}{\Psi'}$ if $\Psi = \Psi'$ or
  $\Psi \searrow^* \Psi'$, where $\searrow^*$ is the transitive
  closure of $\searrow$.
\end{definition}

\begin{lemma}
  $\disassembleSelection{}{}$ is a partial order.
\end{lemma}

\begin{lemma}
  For every selection $\Psi$, there is a unique maximal
  element $\Psi'$ such that $\disassembleSelection{\Psi'}{\Psi}$.
\end{lemma}

\begin{definition}
  \note{define function that returns unique maximal element of selection}
\end{definition}


\subsection{Frames \& zippers}
\begin{figure}
  \vspace{-3px}
  \[\arraycolsep=3pt\begin{array}{rlrl}
    \text{sequence frame} & \tframe^s & ::= & \tframelit{\tiles^s\framehole\tiles^s}{\tfrelem^s} \\
    \text{pattern tile frame} & \tfrelem^{\pat} & ::= &
      \tframelit{\sparen{\framehole}}{\tframe^{\pat}} \\
    & & \vert &
      \tframelit{\tlam{\framehole}}{\tframe^{\expr}} \\
    & & \vert &
      \tframelit{\tlet{\framehole}{\tiles^{\expr}}}{\tframe^{\expr}} \\
    \text{expression tile frame} & \tfrelem^{\expr} & ::= &
      \froot \\
    & & \vert &
      \tframelit{\sparen{\framehole}}{\tframe^{\expr}} \\
    & & \vert &
      \tframelit{\tlet{\tiles^{\pat}}{\framehole}}{\tframe^{\expr}}
      % \scond{}{\tiles^{\expr}}{}
  \end{array}\]
  \caption{
    Syntax of pattern and expression frames
  }
  \label{fig:frame-syntax}
\end{figure}


\subsubsection{Actions}
\note{
  Define actions more generally than \tylr~ implementation,
  such that delimiters are assigned sorts to their ends,
  selections are not restricted to shards and
  same-sort tiles, and only selections whose ends are the
  same may enter restructuring mode.
}

% (2 * [3])  ->  (2 * (3 + [_]))

\begin{itemize}
  \item explain action judgment + choice rules
  \item theorems
  \begin{itemize}
    \item movability
    \item selectability
    \item restructuring is sound and complete
  \end{itemize}
\end{itemize}

\subsection{Typechecking}
\note{
  Define tree-structured variants of tiles and frames.
  Define precedence parser that converts linear forms to tree forms
  (or assume existence of such parser?).
  Define type
}
