\section{Overview}\label{sec:overview}

Tiles-based structure editing departs from traditional
structure editors in avoiding direct user modifications
to the AST.
Instead, much like with a text editor, the user modifies a separate
``flattened'' editing structure that affords more flexible
editing mechanisms, which is subsequently parsed into the
abstract syntax.
Unlike a text editor, however, a tiles-based editor ensures
that the edit state can always be successfully parsed.

In this section we motivate and give an example-driven
overview of tiles-based editing as implemented
in \tylr. We begin by describing the limitations of
traditional structure editors, then show how tiles-based
editing overcomes these limitations.


\begin{figure}
  \vspace{-3px}
  \[
  \arraycolsep=3pt\begin{array}{rlrl}
      \mathsf{Type} & \styp & ::= &
          % \tnum ~\vert~
          \shole ~\vert~
          \sbool ~\vert~
          \snum ~\vert~
          \sarr{\styp}{\styp} ~\vert~
          \sprod{\styp}{\styp}\\
      \mathsf{Pattern} & \spat & ::= &
        \shole ~\vert~
        \svar{x} ~\vert~
        \sann{p}{\styp} ~\vert~
        \sprod{\spat}{\spat}\\
      \mathsf{Expression} & \sexp & ::= &
        \shole ~\vert~
        \sboollit{b} ~\vert~
        \snumlit{n} ~\vert~
        \svar{x} \\
      & & \vert &
        \slam{\spat}{\sexp} ~\vert~
        \slet{\spat}{\sexp}{\sexp}\\
      & & \vert &
        \sap{\sexp}{\sexp} ~\vert~
        \splus{\sexp}{\sexp} ~\vert~
        \sequals{\sexp}{\sexp} ~\vert~
        \scond{\sexp}{\sexp}{\sexp}
  \end{array}\]
  \caption{
    Abstract syntax of types $\styp$, patterns $\spat$, and expressions $\sexp$.
    Here, $x$ ranges over variables, $b$ over boolean values, and $n$ over natural numbers.
  }
  \label{fig:language-syntax}
\end{figure}


\subsection{A typical structure editor} \label{sec:simple-editor}
\note{
  Start by describing running language example
  and a simple structure editor for it, which
  despite its simplicity is representative of
  contemporary editors.
  Note things it can't do that you can do in a text
  editor and should be able to do.
}

\note{consider giving names to each of these problems}

\begin{itemize}
\item
  we now describe a few classes of edits that we may wish to perform---and
  could perform in a text editor---but cannot with the given structured
  interface
\item structure-oblivious linear construction of operator sequences
  \begin{itemize}
    \item eg going from \texttt{2 * 3 + 4} as opposed to \texttt{2 * (3 + 4)}
    \item this particular limitation of naive structure editing has received
      the most attention in prior work
    \begin{itemize}
      \item some structure editors defer to text at the leaves
      \item others develop more sophisticated methods, e.g.,
        modeless structure editing article,
        MPS's side transforms
    \end{itemize}
  \end{itemize}
\item deleting the root of a term,
  leaving behind its children for further processing
  (eg splicing into the surrounding context)
  \begin{itemize}
    \item eg it is not possible to delete a let expression and leave behind its body
    \item eg it is not possible to remove a conditional expression
      and leave behind a branch to take its place, or to subsequently join the
      two remaining branches with an operator
    \item note how there's no room in the strict tree structure to deal with
      multiple "floating" children
    \item MPS mitigates this by leaving behind its first child if its the
      same sort, but already this does not satisfy the use case described above,
      and in general the user should have the freedom to choose
  \end{itemize}
\item selecting and restructuring sub- and cross-structural
  parts of the program
  \begin{itemize}
    \item eg it is not possible to select \texttt{3 + 4} in \texttt{2 * 3 + 4}
      and re-associate the expression, as one might in a text editor by
      wrapping the selection in parentheses
    \item eg it is not possible to select \texttt{let x = \_ in} and
      move it before \texttt{let y = 2 in}
    \item eg it is not possible to select \texttt{in} of \texttt{let x = \_ in}
      and move it to wrap the subsequent portion of the expression
    \begin{itemize}
      \item as one might in a text editor by deleting \texttt{in},
        moving the caret, re-typing it elsewhere
      \item as one might when constructing the let line for the first time:
        having typed out \texttt{let x =}, it remains to move the caret over
        and type the \texttt{in}
    \end{itemize}
    \item while some of these examples are contrived given the simplicity of
      the language, such selections and edits occur frequently in practice
    \begin{itemize}
      \item eg 6\% of logged edits in Design Requirements
        paper involve making
        selections across structural boundaries, 10\% of edits excluding those
        that only involve name changes rather than structural changes
    \end{itemize}
  \end{itemize}
\end{itemize}

\subsection{\tylr: a tiles-based structure editor}

We now describe the design of \tylr, a tiles-based structure
editor.
Given the disadvantages of operating directly on abstract syntax
trees, \tylr~ presents programs to the user in a separate
concrete syntax equipped with its own syntax-directed editing
mechanisms.
The concrete syntax is a ``flattened'' version of the abstract
syntax, where the structural units correspond not to semantic
terms in the language but rather syntactic groups of matching
delimiters and their bidelimited children.
We call these structural units \emph{tiles}.
Figure \ref{fig:tile-syntax} shows the subset of \tylr's
concrete syntax corresponding to the abstract syntax in
Figure \ref{fig:language-syntax}.

Like a text editor, \tylr~ features a caret that resides between
tiles...

\subsubsection{Linear construction of operator sequences}
Figure \note{todo} shows the construction of the operator
sequence \texttt{2 * 3 + 4}. Unlike with the simple editor described
in Section \ref{sec:simple-editor}, it is not necessary

Unlike terms in the abstract syntax, tiles may be arranged
sequentially as well as hierarchically.

Rather than manipulating structures of the language syntax,
a tiles-based structure editor works within a parallel editor syntax.
The central form in the editor syntax is the unassociated operator
sequence. Operator sequence elements each take one of four shapes---operand,
unary prefix, unary postfix, and
binary infix---which we collectively refer to as \emph{tiles}.
Tiles may in turn contain nested operator sequences.

Like text, the editor syntax provides a flattened, more linear representation
of the language syntax.
Unlike text, the editor syntax maintains hierarchies of
bidelimited children and can always be parsed into the
language syntax, provided that the structure first undergoes
a hole fixing pass in which holes are inserted and removed
as needed.

\note{talk about automatic hole fixing + operand vs operator holes}

\begin{figure}
  \vspace{-3px}
  \[
  \arraycolsep=3pt\begin{array}{rlrl}
      \mathsf{Tiles}^s & \tiles^s & ::= & \tile^s_1\dots\tile^s_n\ \ \text{($n \geq 1$)} \\
      \mathsf{Tile}^{\typ} & \tile^{\typ} & ::= &
          % \tnum ~\vert~
          \shole ~\vert~
          \sbool ~\vert~
          \snum ~\vert~
          \sarr{}{} ~\vert~
          \sprod{}{} ~\vert~
          \sparen{\tiles^{\typ}}\\
      \mathsf{Tile}^{\pat} & \tile^{\pat} & ::= &
        \shole ~\vert~
        \svar{x} ~\vert~
        \sann{}{\tiles^{\typ}} ~\vert~
        \sprod{}{} ~\vert~
        \sparen{\tiles^{\pat}}\\
      \mathsf{Tile}^{\expr} & \tile^{\expr} & ::= &
        \shole ~\vert~
        \sboollit{b} ~\vert~
        \snumlit{n} ~\vert~
        \svar{x} ~\vert~
        \sap{}{} ~\vert~
        \splus{}{} ~\vert~
        \sequals{}{} ~\vert~
        \sparen{\tiles^{\exp}} \\
      & & \vert &
        \slam{\tiles^{\pat}}{} ~\vert~
        \slet{\tiles^{\pat}}{\tiles^{\expr}}{} ~\vert~
        \scond{}{\tiles^{\expr}}{}
  \end{array}\]
  \caption{
    Tile syntax \note{todo}
  }
  \label{fig:tile-syntax}
\end{figure}


\note{after describing, note bonus edits that combine limitations from previous selection}

\note{do get into how restructuring mode fits naturally within the deletion vs construction}