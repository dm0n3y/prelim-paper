\section{Implementation}\label{sec:implementation}

\tylr~ is implemented in ReasonML using \texttt{Incr\_dom} \cite{incr-dom},
a model-view-update library, and compiled to JavaScript using
\texttt{Js\_of\_ocaml} \cite{DBLP:conf/aplas/RadanneVB16}.
The total implementation comes out to roughly 6000 lines
of ReasonML code.
The core editing semantics contributes about 1700 lines;
a majority, about 3800 lines, is responsible for frontend
concerns such as measuring and visually decorating code
elements; and the rest are utilities.

We have also implemented a version of \tylr~ that closely adheres
to the restricted calculus sketched in Section \ref{sec:formalism-2}.
In that implementation, the editing logic reduces in size to
about 1200 lines.

The implementation is publicly available at
\url{https://github.com/hazelgrove/tylr}.
Interested readers may visit \url{tylr.fun} to play with a
running instance of the code.





% \note{
%   Talk about \texttt{tylr} design/implementation choices that deviate
%   from and apply restrictions to formalism.
%   Avoid major screenshots, show these instead in Section 2.
%   See if it's possible to avoid Design in section title.
%   Include details about coding details.
% }
% \begin{itemize}
%   \item showing tree structure in pointing mode
%   \begin{itemize}
%     \item parsing as you go
%   \end{itemize}
%   \item no sort dependency cycle
%   \begin{itemize}
%     \item greedy selection
%     \item tile design: same-sort vs diff-sort children
%   \end{itemize}
%   \item picking up multiple matching selections in restructuring mode
%   \begin{itemize}
%     \item copy and pasting of root nodes rather than whole terms
%     \item just as a whole selection can be moved anywhere, matching selections
%       can be moved anywhere as a group
%   \end{itemize}
%   \item enabling restructuring across sorts by unsorting selections
%   \item entering restructuring mode automatically on deletion/construction
% \end{itemize}