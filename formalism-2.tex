\section{Tile-based Editor Calculus}\label{sec:formalism-2}

We now precisely characterize tile-based editing
as a calculus called \ty.
The punchline of \ty~ is the action judgment
$\performAction{\editState_1}{\action}{\editState_2}$,
which sends edit state $\editState_1$ to $\editState_2$
via action $\action$, and its governing invariant
that actions preserve parsability into a well-formed
program term.

We will begin our presentation in Section ??
by describing the syntax of edit states $Z$.
Then, in Section ??, we will present each rule
defining the action judgment
$\performAction{\editState_1}{\action}{\editState_2}$,
describing supporting judgments and their metatheory
as needed along the way.

\subsection{Zippers}

Figure ?? shows the syntax of \ty~ edit states $\zip$,
which follow a variant of the well-known
\emph{zipper} pattern \cite{zipper}.
In particular, they have a \emph{bottom-up} zipper
structure---decomposing the total edit state $\zip$ into a
focused substructure $\subject$ and its surrounding context
$\zframe$---unlike the top-down structure used in other
work \note{cite}.
In this work, we refer to $B$ as the \emph{subject} and $M$
as the \emph{frame}.

\paragraph{Subjects}
Subjects $\subject$ are \emph{modal}, the three
variants corresponding to the pointing, selecting,
and restructuring modes we described using \tylr~
in Section \ref{sec:overview}.
Each mode decomposes the total subject into two
or three \emph{segments} $\selection$.
This decomposition encodes text-like cursor
placement between syntactic entities.
For example, consider the following edit state
in pointing mode in \tylr.

\note{img}

\noindent
The corresponding subject in \ty~ is
\note{(need to figure out under/overlines)}
\[
  \pointing{\optile{\snumlit{1}}\bintile{\tprod}}{\optile{\snumlit{2}}}
\]

Segments $\selection$ are sequences of \emph{tiles}
and \emph{shards}, collectively called \emph{pieces},
as shown in Figure \ref{fig:segment-syntax}.
The two segments $\optile{\snumlit{1}}\bintile{\tprod}$ and
$\optile{\snumlit{2}}$ in the example above consist
entirely of expression tiles $\tile^{\expr}$,
the syntax for which was given in Figure \ref{fig:tile-syntax}.
\note{give another tylr example with a selection containing
a shard, introduce shards in Figure \ref{fig:shard-syntax-2},
give tylrcore version of the edit state in selecting mode}


\begin{figure}
  \vspace{-3px}
  \[
  \arraycolsep=3pt\begin{array}{rlrl}
      \mathsf{EditState} & \editState & ::= & \zip^{\pat} ~\vert~ \zip^{\expr} \\
      \mathsf{Zipper} & \zip^s & ::= & \zipper{\subject}{\tfrelem^s} \\
      % \mathsf{Tile}^{\typ} & \tile^{\typ} & ::= &
      %     % \tnum ~\vert~
      %     \shole ~\vert~
      %     \sbool ~\vert~
      %     \snum ~\vert~
      %     \sarr{}{} ~\vert~
      %     \sprod{}{} ~\vert~
      %     \sparen{\tiles^{\typ}}\\
      \mathsf{Subject} & \subject & ::= &
        \pointing{\selection}{\selection} ~\vert~
        \selecting{\selection}{\selection}{\selection} ~\vert~
        \restructuring{\selection}{\selection}{\selection}
  \end{array}\]
  \caption{
    Syntax of zippers \note{find different notation for selections/restructuring, explain directionality/minimality}
  }
  \label{fig:zipper-syntax}
\end{figure}

\begin{figure}
\[\arraycolsep=3pt\begin{array}{rlrl}
  \text{segment} & \selection & ::= &
  \selem_1\dots\selem_n \\
  \text{piece} & \selem & ::= &
    \tile ~\vert~
    \shard \\

  \text{tile} & \tile & ::= &
      \tile^{\pat} ~\vert~
      \tile^{\expr} \\

  \text{shard} & \shard & ::= &
    \shard^{\pat} ~\vert~
    \shard^{\expr}
\end{array}\]
\caption{
  Syntax of segments
}
\label{fig:segment-syntax}
\end{figure}
\begin{figure}
  \[\arraycolsep=4pt\begin{array}{rlrl}
    \text{pattern shard} & \shard^{\pat} & ::= &
      \shardlit{\texttt{(}} ~\vert~
      \shardlit{\texttt{)}} \\
    \text{expression shard} & \shard^{\expr} & ::= &
      \shardlit{\texttt{(}} ~\vert~
      \shardlit{\texttt{)}} ~\vert~
      \shardlit{\lambda} ~\vert~
      \shardlit{\texttt{.}} ~\vert~
      \shardlit{\texttt{let}} ~\vert~
      \shardlit{\texttt{=}} ~\vert~
      \shardlit{\texttt{in}}
  \end{array}\]
  \caption{Syntax of pattern and expression shards}
  \label{fig:shard-syntax-2}
\end{figure}

\begin{figure}
  \vspace{-3px}
  \[\arraycolsep=3pt\begin{array}{rlrl}
    \text{sequence frame} & \tframe^s & ::= & \tframelit{\tiles^s\framehole\tiles^s}{\tfrelem^s} \\
    \text{pattern tile frame} & \tfrelem^{\pat} & ::= &
      \tframelit{\sparen{\framehole}}{\tframe^{\pat}} \\
    & & \vert &
      \tframelit{\tlam{\framehole}}{\tframe^{\expr}} \\
    & & \vert &
      \tframelit{\tlet{\framehole}{\tiles^{\expr}}}{\tframe^{\expr}} \\
    \text{expression tile frame} & \tfrelem^{\expr} & ::= &
      \froot \\
    & & \vert &
      \tframelit{\sparen{\framehole}}{\tframe^{\expr}} \\
    & & \vert &
      \tframelit{\tlet{\tiles^{\pat}}{\framehole}}{\tframe^{\expr}}
      % \scond{}{\tiles^{\expr}}{}
  \end{array}\]
  \caption{
    Syntax of pattern and expression frames
  }
  \label{fig:frame-syntax}
\end{figure}


\paragraph{Frames}
\emph{Frames} model the rest of the edit state not included
in the subject.
At a high level, frames may be understood as a
bottom-up sequence of nested containers, where
the head container immediately
wraps the subject, and each container points to its
parent container, up to the edit state root.

The syntax for frames is given in Figure \ref{fig:frame-syntax}.
At the top level, frames $\zframe$ are sort-indexed \emph{tile frames}
$\tfrelem^s$, which represent tiles with a missing child of
sort $s$.
Other than the program root $\froot$, we annotate tile frame literals
with $\framehole$ symbols in place of the missing children.
Each tile frame $\tfrelem^s$ points to its parent
\emph{sequence frame} $\tframe^{s'}$, which carries the tile frame's
sibling tiles of sort $s'$.
Each sequence frame $\tframe^s$ in turn points to its parent
tile frame $\tfrelem^s$.
\note{review both examples from subject section}


\subsection{Actions}

We now describe each of the eight rules defining the
action judgment $\performAction{\editState_1}{\action}{\editState_2}$.

\note{introduce action syntax}

\note{remark on first two rules taking the longest to establish setup,
after which remaining six rules build on top of that}

\begin{figure}
  \vspace{-3px}
  \[
  \arraycolsep=3pt\begin{array}{rlrl}
      \text{action} & \action & ::= &
        \actionlit{mark} ~\vert~
        \actionlit{move}~\direction ~\vert~
        \actionlit{insert}~\tile ~\vert~
        \actionlit{remove} \\
      \text{direction} & \direction & ::= &
        \texttt{left} ~\vert~
        \texttt{right}
  \end{array}\]
  \caption{Syntax of actions}
  \label{fig:action-syntax}
\end{figure}


\subsubsection{Moving}
Our design of zipper operations diverges from prior art
in our pursuit of linear text-like interactivity.
Prior zipper designs are strictly hierarchical and ask
the client to navigate in a structure-aware two-dimensional
fashion: left and right to traverse siblings,
up and down to traverse parent-child relations.
In contrast, we aim to support a one-dimensional interaction
model:
it should be possible to navigate freely by moving solely
left and right \note{might need to clarify or change wording
here to address comparison with ordered traversals of trees};
furthermore it should be possible to specify
arbitrary subranges of a one-dimensional serialization
of the total structure.
Toward these ends, actions in \ty~ rely
on a form of incremental unparsing and reparsing by which
hierarchical structures may be ``flattened'' as needed
to support linear traversal and partial selection.

\[
  \inferrule[Move]{
    \pmove{\zipper{\pointing{\selection_1}{\selection_2}}{\zframe}}{\direction}{\editState}
  }{
    \performAction{
      \zipper{\pointing{\selection_1}{\selection_2}}{\zframe}
    }{
      \actionlit{move}\ \direction
    }{
      \editState
    }
  }
\]


\subsubsection{Constructing}
\[
  \inferrule[Construct]{
    \fixHolesSelections{\selection_1}{\tile^s\selection_2}{s}{\selection'_1}{\selection'_2}
  }{
    \performAction{
      \zipper{\pointing{\selection_1}{\selection_2}}{\tfrelem^s}
    }{
      \actionlit{construct}\ \tile^s
    }{
      \zipper{\pointing{\selection'_1}{\selection'_2}}{\tfrelem^{s}}
    }
  }
\]

\subsubsection{Selecting}
\[
  \inferrule[StartSelecting]{
  }{
    \performAction{
      \zipper{\pointing{\selection_1}{\selection_2}}{\zframe}
    }{
      \actionlit{mark}
    }{
      \zipper{\selecting{\selection_1}{\strut~\cdot~}{\selection_2}}{\zframe}
    }
  }
\]

\[
  \inferrule[Select]{
    \smove{\zipper{\selecting{\selection_1}{\selection_2}{\selection_3}}{\zframe}}{\direction}{\editState}
  }{
    \performAction{
      \zipper{\selecting{\selection_1}{\selection_2}{\selection_3}}{\zframe}
    }{
      \actionlit{move}\ \direction
    }{
      \editState
    }
  }
\]

\subsubsection{Restructuring}

\[
  \inferrule[PickUp]{
    \sortoftip{\leftTip{\selection_2}} = \sortoftip{\rightTip{\selection_2}} \\
    \fixHolesSelections{\selection_1}{\selection_3}{s}{\selection'_1}{\selection'_3}
  }{
    \performAction{
      \zipper{\selecting{\selection_1}{\selection_2}{\selection_3}}{\tfrelem^s}
    }{
      \actionlit{mark}
    }{
      \zipper{\restructuring{\selection'_1}{\selection_2}{\selection'_3}}{\tfrelem^s}
    }
  }
\]

\[
  \inferrule[Restructure]{
    \rmove{\zipper{\restructuring{\selection_1}{\selection_2}{\selection_3}}{\zframe}}{\direction}{\editState}
  }{
    \performAction{
      \zipper{\restructuring{\selection_1}{\selection_2}{\selection_3}}{\zframe}
    }{
      \actionlit{move}\ \direction
    }{
      \editState
    }
  }
\]

\[
  \inferrule[PutDown]{
    \text{\note{not\ }}\wholeSelection{s'}{\selection_2}
    \vee \wholeSelection{s}{\selection_2} \\
    \fixHolesSelections{\selection_1}{\selection_2\selection_3}{s}{\selection_4}{\selection_5}
  }{
    \performAction{
      \zipper{\restructuring{\selection_1}{\selection_2}{\selection_3}}{\tfrelem^s}
    }{
      \actionlit{mark}
    }{
      \parseZipper{\zipper{\pointing{\selection_4}{\parseSelection{\selection_5}}}{\tfrelem^s}}
    }
  }
\]

\subsection{Deleting}
\[
  \inferrule[Delete]{
    \fixHolesSelections{\filterTiles{s}{\selection_1}}{\filterTiles{s}{\selection_3}}{s}{\selection'_1}{\selection'_3}
  }{
    \performAction{
      \zipper{\restructuring{\selection_1}{\selection_2}{\selection_3}}{\tfrelem^s}
    }{
      \actionlit{delete}
    }{
      \zipper{\pointing{\selection'_1}{\selection'_3}}{\tfrelem^s}
    }
  }
\]