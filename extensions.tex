\section{Extensions \& Limitations}

\tylr~ is a minimal demonstration of tile-based
editing, optimized for exposition rather than practical use,
while \ty~ formalizes an essential subset of \tylr's functionality.
In this section, we will sketch how \ty~ may be extended
in simple ways to support \tylr's full feature set,
then discuss our ongoing efforts to scale up tile-based
editing to support larger-scale program authoring.

\subsection{From \ty~ to \tylr}

\ty~ does not model \tylr's full feature set.
In particular, it doesn not support left-to-right selecting,
multi-selection restructuring, or automatic restructuring
upon construction.

The first limitation is easily overcome by adding
a new variant $\selectingRight{\selection}{\selection}{\selection}$
to the subjects $\subject$ in Figure \ref{fig:zipper-syntax}
and extending the auxiliary movement judgment
$\smove{\editState_1}{\direction}{\editState_2}$
in Figure \ref{fig:move-selecting}
with mirror forms of its current rules.

The second and third limitations are related
and are overcome together.
First, the restructuring variant $\restructuring{\selection}{\selection}{\selection}$
of subjects is replaced with $\restructuring{\selection}{\mathcal{C}}\selection$,
where $\mathcal{C}$ is a zippered list of selections in the backpack,
the focused selection being the one to put down next.
Second, the sort-specific shard datatypes in Figure \ref{fig:shard-syntax-2}
are extended with
the new form $\shardlit{\texttt{let}~\tiles^{\pat}~\texttt{=}}$.
More generally, shards are extended with new forms
representing intermediate assemblies of shards that
bidelimit a tile's children of different sort.
Such intermediate shards have the useful property of having same-sort
tips and represent the minimal restructurable units of the
parent tile.
Finally we extend the action datatype in Figure \ref{fig:action-syntax}
so that the $\actionlit{construct}$ action takes a shard instead of
a complete tile---specifically, we expect this shard to have same-sort tips.
Constructing such a shard then enters restructuring mode
upon with the matching same-sort-tipped shards in the backpack.


\subsection{Ongoing work}

\note{talk about evaluation plans}

\begin{itemize}
\item pilot study of earlier ad hoc design, Hazel implementation in progress for user study of latest design
\item parsed structure should be maintained with edit state
\item dealing with text literals
\item empty delimiters
\item sequences eg blocks of statements
\item automatic selection and restructuring
\item rematching delimiters
\item shared delimiters (eg parenthesized vs fn tight ap)
\item 2d layout
\item free floating comments
\end{itemize}