\section{Extensions \& Ongoing Work}

\tylr~ is a minimal demonstration of tile-based
editing, optimized for exposition rather than practical use,
while \ty~ formalizes an essential subset of \tylr's functionality.
In this section, we will sketch how \ty~ may be extended
in simple ways to support \tylr's full feature set,
then discuss our ongoing efforts to scale up tile-based
editing to support larger-scale program authoring.

\subsection{From \ty~ to \tylr}

\ty~ does not model \tylr's full feature set.
In particular, it doesn not support left-to-right selecting,
multi-selection restructuring, or automatic restructuring
upon construction.

The first limitation is easily overcome by adding
a new variant $\selectingRight{\selection}{\selection}{\selection}$
to the subjects $\subject$ in Figure \ref{fig:zipper-syntax}
and extending the auxiliary movement judgment
$\smove{\editState_1}{\direction}{\editState_2}$
in Figure \ref{fig:move-selecting}
with mirror forms of its current rules.

The second and third limitations are related
and are overcome together.
First, the restructuring variant $\restructuring{\selection}{\selection}{\selection}$
of subjects is replaced with $\restructuring{\selection}{\mathcal{C}}\selection$,
where $\mathcal{C}$ is a zippered list of selections in the backpack,
the focused selection being the one to put down next.
Second, the sort-specific shard datatypes in Figure \ref{fig:shard-syntax-2}
are extended with
the new form $\shardlit{\texttt{let}~\tiles^{\pat}~\texttt{=}}$.
More generally, shards are extended with new forms
representing intermediate assemblies of shards that
bidelimit a tile's children of different sort.
Such intermediate shards have the useful property of having same-sort
tips and represent the minimal restructurable units of the
parent tile.
Finally we extend the action datatype in Figure \ref{fig:action-syntax}
so that the $\actionlit{construct}$ action takes a shard instead of
a complete tile---specifically, we expect this shard to have same-sort tips.
Constructing such a shard then enters restructuring mode
upon with the matching same-sort-tipped shards in the backpack.


\subsection{Ongoing work}

\tylr~ demonstrates novel structural selection and restructuring
methods, but as an expository tool, it is limited
as a practical authoring tool in several respects.
\tylr~ edit states are visually rendered as a single line
no matter their size.
Construction is invoked with single keypresses (\eg,
a \code{let}-tile is constructed using the \texttt{=} key)
and number literals and variables are limited to
single digits and characters.
Moreover, \tylr's underlying language is tiny and
features neither a type system nor an evaluation semantics.

We developed \tylr~ with the intention of scaling up
its editing techniques to Hazel \note{cite},
a more mature structure editor that renders
edit states in the usual multi-line text-like layout.
Scaling up tile-based editing to multi-line output
poses new design and engineering challenges
such as integrating with pretty printed layout,
often computationally expensive to produce;
visually summarizing large selections in the backpack;
and animating layout transitions, which
we believe will become necessary to
comprehend the rearrangement of multiple large program
components in the course of restructuring.

Hazel additionally maintains strong
\emph{semantic} guarantees: every edit state
has a well-defined type and can be run to
produce a well-defined result.
The underlying expression language features, in
addition to empty holes in like \tylr, non-empty
holes that cordon off ill-typed portions
of the program.
Hazel is currently designed around
a type-directed action semantics, which
propagates contextual typing information
down to the subject of its edit state
so that it can insert and remove non-empty holes
as needed to maintain well-formedness.
We are currently working on theoretically decoupling
Hazel's typing concerns from the editing concerns
so that we may modularly extend tile-based editing
with Hazel-style semantic invariants.

We implemented and ran a small pilot study on an early limited
prototype of restructuring mode in July 2019, which indicated
users quickly understood and appreciated the workflow.
In the coming months, we plan to run a more thorough, controlled user study in which
we compare the efficiency of and user sentiment
toward text editing and tile-based editing in Hazel.
In the long run, we would like to integrate Hazel into
select assignments of EECS 490, the programming languages
course at the University of Michigan, to evaluate the
usability and effectiveness of tile-based structure editing
in an educational context.
