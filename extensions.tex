\section{Extensions \& Limitations}

\tylr~ is a minimal demonstration of tile-based
editing, optimized for exposition rather than practical use,
while \ty~ formalizes an essential subset of \tylr's functionality.
In this section, we will sketch how \ty~ may be extended
in simple ways to support \tylr's full feature set,
then discuss our ongoing efforts to scale up tile-based
editing to support larger-scale program authoring.

\subsection{From \ty~ to \tylr}

\ty~ does not model \tylr~ full feature set.
In particular, it doesn not support left-to-right selecting
multi-selection restructuring, or automatic restructuring
upon construction.

The first limitation is easily overcome by adding
a new variant $\selectingRight{\selection}{\selection}{\selection}$
to the subjects $\subject$ in Figure \ref{fig:zipper-syntax}
and extending the auxiliary movement judgment
$\smove{\editState_1}{\direction}{\editState_2}$
in Figure \ref{fig:move-selecting}
with mirror forms of its current rules.

The second and third limitations are related
and can be overcome together.
This involves extending
the restructuring variant $\restructuring{\selection}{\selection}{\selection}$...
\note{talk about extending backpack to be zippered list
carrying multiple selections, extending intact vs cracked
check in restructuring rules to full vs hungry,
and adding intermediate same-sort-ended shard variants}


\subsection{Ongoing work}

\note{talk about evaluation plans}

\begin{itemize}
\item pilot study of earlier ad hoc design, Hazel implementation in progress for user study of latest design
\item parsed structure should be maintained with edit state
\item dealing with text literals
\item empty delimiters
\item sequences eg blocks of statements
\item automatic selection and restructuring
\item rematching delimiters
\item shared delimiters (eg parenthesized vs fn tight ap)
\item 2d layout
\item free floating comments
\end{itemize}